\section{Proof of Theorem~\ref{thm:ciq_convergence}}
\label{app:ciq_proofs}

To prove the convergence result in \cref{thm:ciq_convergence}, we first prove the following lemmas.

\begin{lemma}
  Let $\bK \succ 0$ be symmetric positive definite and let shifts $t_1$, $\ldots$, $t_Q > 0$ be real-valued and positive.
  After $J$ iterations of msMINRES, all shifted solve residuals are bounded by:
  %
  \begin{align*}
    \bigl\Vert (\bK + t_q \bI) \bc_J^{(q)} - \bb \bigr\Vert_2
    &\leq \left( \frac{
      \sqrt{\kappa(\bK + t_q \bI)} - 1
    }{
      \sqrt{\kappa(\bK + t_q \bI)} + 1
    }\right)^J
    \Vert \bb \Vert_2
    \leq \left( \frac{
      \sqrt{\kappa(\bK)} - 1
    }{
      \sqrt{\kappa(\bK)} + 1
    }\right)^J
    \Vert \bb \Vert_2,
	\end{align*}
  %
  where $\bb$ is the vector to solve against, $\bc^{(1)}_J$, $\ldots$, $\bc^{(Q)}$ are the msMINRES outputs, and $\kappa(\bK)$ is the condition number of $\bK$.
  \label{lemma:minres}
\end{lemma}
%
\begin{proof}
  The convergence proof uses a polynomial bound, which is the standard approach for Krylov algorithms.
  %\footnote{
    %\citet{paige1975solution} do not provide a detailed convergence analysis of MINRES.
    %However, proving the following result is a straightfoward analog of the famous conjugate gradients convergence analysis.
  %}
  See \citep[e.g.][]{shewchuk1994introduction,trefethen1997numerical,saad2003iterative} for an analogous proof for the conjugate gradients method.

	At iteration $J$, the msMINRES algorithm produces:
  %
	\begin{align}
    \bc^{(q)}_J
    = \argmin_{\bc^{(q)} \in \mathcal{K}_J(\bK, \bb)} \Bigl[
      \bigl\Vert (\bK + t_q \bI) \bc^{(q)} - \bb \bigr\Vert_2
    \Bigr],
    \quad
    q \in [1, Q],
    \label{eqn:minres_krylov}
	\end{align}
  %
  where without loss of generality we assume $\bc_0^{(q)} = \bzero$ for simplicity.
  Since the Krylov subspace $\mathcal{K}^{(J)} (\bK, \bb) = \mathcal{K}^{(J)} (\bK + t_q \bI, \bb)$ contains powers of $\bK + t_q \bI$ applied to the vector $\bb$, we can rewrite
  %
  $\bc_j^{(q)}$
  as a polynomial of $(\bK + t_q \bI)$ applied to $\bb$:
  %
  \begin{align*}
    \bc^{(q)}_J
    &= \sum_{j=0}^{J-1} \nu_j^{(q)} (\bK + t_q \bI)^j \bb, \quad \nu_j \in \reals
    \\
    &= P_q(\bK + t_q \bI) \bb, \quad P_q(\cdot) \in \mathcal{P}_{J-1}
  \end{align*}
  %
  where $\nu_0$, $\ldots$, $\nu_J$ are some coefficients,
  and $\mathcal{P}_{J-1}$ is the class of $(J-1)$-degree polynomials.
  Consequentially, the residual can be written as:
  %
  \begin{align}
    (\bK + t_q \bI) \bc^{(q)} - \bb
    &= \bigl[ \left( \bK + t_q \bI \right) P_q(\bK + t_q \bI) - \bI \bigr] \bb, \quad P_q(\cdot) \in \mathcal{P}_{J-1}
    \nonumber
    \\
    &= P'_q(\bK + t_q \bI) \bb, \quad P'_q(\cdot) \in \widetilde{\mathcal{P}_J}
    \nonumber
  \end{align}

  where $\widetilde{\mathcal{P}_J}$ is the class of $J$-degree polynomials that additionally satisfy the condition $P'_q(0) = -1$ for some $\xi$.
  Therefore, \cref{eqn:minres_krylov} is equivalent to an optimal polynomial problem:
	\begin{align}
    %\min_{\bc^{(q)} \in \mathcal{K}_J(\bK, \bb)} \Bigl[
      %\bigl\Vert (\bK + t_q \bI) \bc^{(q)} - \bb \bigr\Vert_2
    %\Bigr]
    \bigl\Vert (\bK + t_q \bI) \bc_J^{(q)} - \bb \bigr\Vert_2
    &= \min_{P'_q \in \widetilde{\mathcal{P}_J}} \Bigl[
      \Vert P'_q(\bK) \bb \Vert_2
    \Bigr].
    \label{eqn:minres_poly}
  \end{align}
  %
  We can upper bound \cref{eqn:minres_poly} by considering the case where $\bb$ is the ``worst'' eigenvector of $\bK$:
  %
  \begin{align}
    \bigl\Vert (\bK + t_q \bI) \bc_J^{(q)} - \bb \bigr\Vert_2
    &\leq \min_{P'_q \in \mathcal{P}^{(t_q)}_J} \Bigl[
      \max_{\lambda \in \Lambda}
      \left\vert P'_q(\lambda + t_q) \right\vert
    \Bigr] \Vert \bb \Vert_2,
    \quad
    q \in [1, Q].
    \label{eqn:bound}
	\end{align}
  %
  where $\Lambda$ is the set of eigenvalues of $\bK$.
  Here we exploit the fact that $\bK + t_q \bI$ and $\bK$ share the same eigenvectors and their eigenvalues are shifted by $t_q$.

  We can replace the minimum in \cref{eqn:bound} with any $J^\text{th}$ degree polynomial $P'(\cdot)$ that satisfies $P'(0) = -1$.
  In Krylov subspace proofs, it is common to construct such a $P'(\cdot)$ using a ratio of shifted Chebychev polynomials (see \citep[e.g.][Sec. 9.2]{shewchuk1994introduction} or \citep[e.g.][Thm. 38.5]{trefethen1997numerical} for a detailed derivation).
  With this choice of polynomial
  %
  \begin{equation}
    P'(\cdot)
    \leq \left( \frac{
      \sqrt{\kappa(\bK + t_q \bI)} - 1
    }{
      \sqrt{\kappa(\bK + t_q \bI)} + 1
    }\right)^J.
    \label{eqn:chebychev}
  \end{equation}
  Note that as $t_q > 0$ the conditioning of $\kappa(\bK + t_q \bI)$ decreases and so trivially we have that
  %
  \begin{equation}
    \frac{
      \sqrt{\kappa(\bK + t_q \bI)} - 1
    }{
      \sqrt{\kappa(\bK + t_q \bI)} + 1
    }
    \leq
    \frac{
      \sqrt{\kappa(\bK)} - 1
    }{
      \sqrt{\kappa(\bK)} + 1
    }.
    \label{eqn:chebychev_upper}
  \end{equation}
  Plugging \cref{eqn:chebychev,eqn:chebychev_upper} into \cref{eqn:bound} as an upper bound on $P_q(\cdot)$ completes the proof.
\end{proof}

\cref{lemma:minres} is a very loose bound, as it doesn't assume anything about the spectrum of $\bK$.
In practice, we find that smMINRES converges for many covariance matrices with $J \approx 100$, even when the conditioning is on the order of $\kappa(\bK) \approx 10^4$.
This convergence is faster with preconditioning.




%%%%
\begin{lemma}
  \begin{align}
    \frac{
      \sqrt{\kappa(\bK + t_q \bI)} - 1
    }{
      \sqrt{\kappa(\bK + t_q \bI)} + 1
    } = \frac{\sqrt{\lambda_\text{max} + t} - \sqrt{\lambda_\text{min} + t}  }{\sqrt{\lambda_\text{max} + t} + \sqrt{\lambda_\text{min} + t}  }
    < \frac{\lambda_\text{max}}{4t}
  \end{align}
  \label{lemma:condition}
\end{lemma}
\begin{proof}
  \gp{TODO} use the bound $\sqrt{x+1} - \sqrt{x} < \frac{1}{2 \sqrt{x}}$.
\end{proof}
%%%%




\begin{lemma}
  Let $\sigma_q^2$ and $\widetilde w_q$ be defined as in \cref{eqn:quad_points_and_locations}.
  Then
  \begin{equation*}
    \sum_q \frac{|w_q|}{|t_q|} = \sum_q \frac{|\widetilde w_q|}{|\sigma^2_q|} < \frac{4  \log \left( 5 \sqrt{\kappa(K)} \right)  }{\pi}Q\sqrt{\lambda_\text{min}} \\
  \end{equation*}
  where $w_q = -\widetilde w_q$ and $t_q = -\sigma^2_q$ as used in \cref{eqn:contour_integral_quad_4}.
  \label{lemma:quad_ratio}
\end{lemma}

\begin{proof}
  Using facts about elliptical integrals and functions, we have:
  \begin{align}
    \mathcal{K}'(k) < \log(1 + 4 / k) \leq \log( 5 / k) \qquad {\rm for} \qquad k \in [0, 1]
    \tag{From \citep[][Thm. 2]{yang2019convexity}}
    \\
    \text{dn}(u \mathcal{K}(k) | k ) \le \frac{\text{sn}(u \mathcal{K}(k) | k )^2}{\sin^2(\pi u/2)} \qquad {\rm for} \qquad 0 \le u \le 1
    \tag{From \gp{? maybe \citep{siegl2016extremal}}}
    \\
    \sum_q \frac{1}{ Q \sin^2 \frac{\pi(q-\tfrac{1}{2})}{2Q} }  = 2 Q
    \tag{From \gp{?}}
  \end{align}
  %
  Then,
  \begin{align*}
    \frac{|w_q|}{|t_q|}
    =
    \frac{|\widetilde w_q|}{|\sigma^2_q|}
    &= \frac{
      \frac{ 2 }{ \pi Q \sqrt{\lambda_\text{min}} }
      \:\: \left[
      \mathcal{K}'( k )
      \:\:\: \text{cn} \left( i u_q \mathcal{K}'(k) \mid k \right)
      \:\:\: \text{dn} \left( i u_q \mathcal{K}'(k) \mid k \right)
      \right]
    }{
      \left| \lambda_\text{min}^{-1} \Bigl( \text{sn}(i u_q \mathcal{K}'(k) \mid k) \Bigr)^2 \right|
    }
    \\
    &= \frac{
      \frac{ 2 }{ \pi Q \sqrt{\lambda_\text{min}} }
      \:\: \left[
      \mathcal{K}'( k )
      \frac{
        1
      }{
        \text{cn} \left( u_q \mathcal{K}'(k) \mid k' \right)
      }
      \frac{
        \text{dn} \left( u_q \mathcal{K}'(k) \mid k' \right)
      }{
        \text{cn} \left( u_q \mathcal{K}'(k) \mid k' \right)
      }
      \right]
    }{
      \left| \lambda_\text{min}^{-1} \Bigl( i \frac{
        \text{sn}(u_q \mathcal{K}'(k) \mid k')
      }{
        \text{cn}(u_q \mathcal{K}'(k) \mid k')
      }\Bigr)^2 \right|
    }
    \tag{via Jacobi imaginary transform}
    \\
    &\leq \frac{
      \frac{ 2 }{ \pi Q \sqrt{\lambda_\text{min}} }
      \:\: \left[
        \mathcal{K}'( k ) \frac{
          ({
            \text{sn}(u_q \mathcal{K}'(k) \mid k')
          })^2
        }{
          \sin^2 ( \pi u_q / 2)
        }
      \right]
    }{
      \lambda_\text{min}^{-1} \Bigl({
        \text{sn}(u_q \mathcal{K}'(k) \mid k')
      }\Bigr)^2
    }
    \\
    &= \frac{ 2 \sqrt{\lambda_\text{min}} }{ \pi Q \sin^2 ( \pi u_q / 2) } \mathcal{K}'(k)
    \\
    &< \frac{ 2 \sqrt{\lambda_\text{min}} \log(5 / k) }{ \pi Q \sin^2 ( \pi u_q / 2) }
  \end{align*}
  %
  Plugging in the values of $k$, $u_q$ and summing over $u_q$ values gives us
  %
  \begin{align}
    \sum_{q} \frac{|w_q|}{|t_q|}
    &<
    \sum_{q} \frac{ 2 \sqrt{\lambda_\text{min}} \log \left( 5 \sqrt{\kappa(\bK)} \right)}
    { \pi Q \sin^2 ( \frac{\pi (q - 1/2}{2Q}) }
    = \frac{4 \log \left( 5 \sqrt{\kappa(\bK)} \right)}{\pi} Q \sqrt{\lambda_\text{min}}.
  \end{align}
\end{proof}







With these lemmas we are able to prove our primary CIQ convergence result:
%
\newtheorem*{ciq_convergence}{Theorem~\ref{thm:ciq_convergence} (Restated)}
\begin{ciq_convergence}
  Let $\bK$ and $\bb$ be the inputs to CIQ, producing the output $\bv_J \approx \bK^{1/2} \bb$ after $J$ iterations of msMINRES with $Q$ quadrature points.
  The difference between $\bv_J$ and $\bK^{1/2} \bb$ is bounded by:
  %
  \begin{align*}
    \left\Vert \ba_J - \bK^{\frac 1 2} \bb \right\Vert_2
    &\leq
    \overbracket{
      \bigo{\exp\left( -\frac  {2 Q \pi^2}{\log \kappa(\bK) + 3} \right)}
    }^{\text{Quadrature error}}
    \\
    &\phantom{=} +
    \overbracket{
      \frac{ 2 \log \left( 5 \sqrt{\kappa(\bK)} \right)  \left( Q \lambda_\text{max} \sqrt{\lambda_\text{min}} \right) }{\pi}
      \left( \frac{ \sqrt{\kappa(\bK)} - 1}{ \sqrt{\kappa(\bK)} + 1} \right)^{J-1}
      \left\Vert \bb \right\Vert_2.
    }^{\text{msMINRES error}}
  \end{align*}
  %
  where $\lambda_\text{max},\lambda_{\text{min}}$ are the max and min eigenvalues of $\bK$, and $\kappa(\bK)$ is the condition number of $\bK$.
\end{ciq_convergence}
%
\begin{proof}
  First we note that the CIQ solution $\ba_J$ can be written as $\sum_{i=1} w_q \bc^{(q)}_J$, where $\bc^{(q)}_J$ is the $q^\text{th}$ shifted solve $\approx (t_q \bI + \bK)^{-1} \bb$ from msMINRES.
  Applying the triangle inequality we have:
  %
  \begin{align}
    \left\Vert \ba_J - \bK^{\frac 1 2} \bb \right\Vert_2
    &=
    \left\Vert \overbracket{\sum_{q=1}^Q w_q \bc^{(q)}_J - \left( \bK \sum_{q=1}^Q w_q \left( t_q \bI + \bK \right)^{-1} \right) \bb }^{\text{msMINRES error}} \right.
    \nonumber
    \\
    &\phantom{=} \quad \left. + \underbracket{\left( \bK \sum_{q=1}^Q w_q \left( t_q \bI + \bK \right)^{-1} \right) \bb - \bK^{\frac 1 2} \bb}_{\text{Quadrature error}} \right\Vert_2
    \nonumber
    \\
    &\leq \sum_{q=1}^Q \vert w_q \vert \left\Vert \bc^{(q)}_J - \bK \left( t_q \bI + \bK \right)^{-1} \bb \right\Vert_2
    \nonumber
    \\
    &\phantom{=} \:\: + \left\Vert \bK \left( \sum_{q=1}^Q w_q \left( t_q \bI + \bK \right)^{-1} \right) \bb - \bK^{\frac 1 2} \bb \right\Vert_2
    \label{eqn:pre_bound}
  \end{align}
  %
  Plugging \cref{lemma:minres} into the msMINRES bound, we have:
  \begin{align*}
    &2 \sum_{q=1}^Q \left\vert w_q \right\vert
    \left( \frac{ \sqrt{\kappa(\bK + t_q \bI)} - 1}{ \sqrt{\kappa(\bK + t_q \bI)} + 1} \right)^J \left\Vert \bb \right\Vert_2
    \\
    \leq \:\:
    &2 \sum_{q=1}^Q \left\vert w_q \right\vert
    \left( \frac{ \sqrt{\kappa(\bK + t_q \bI)} - 1}{ \sqrt{\kappa(\bK + t_q \bI)} + 1} \right)
    \left( \frac{ \sqrt{\kappa(\bK)} - 1}{ \sqrt{\kappa(\bK)} + 1} \right)^{J-1}
    \left\Vert \bb \right\Vert_2
    \tag{via \eqref{eqn:chebychev_upper}}
    \\
    \leq \:\:
    &2 \sum_{q=1}^Q \left\vert w_q \right\vert
    \left( \frac{ \lambda_\text{max} }{ 4 t_q } \right)
    \left( \frac{ \sqrt{\kappa(\bK)} - 1}{ \sqrt{\kappa(\bK)} + 1} \right)^{J-1}
    \left\Vert \bb \right\Vert_2
    \tag{via \cref{lemma:condition}}
    \\
    \leq \:\:
    & \frac{ 2 \log \left( 5 \sqrt{\kappa(\bK)} \right) }{\pi} \left( Q \lambda_\text{max} \sqrt{\lambda_\text{min}} \right)
    \left( \frac{ \sqrt{\kappa(\bK)} - 1}{ \sqrt{\kappa(\bK)} + 1} \right)^{J-1}
    \left\Vert \bb \right\Vert_2.
    \tag{via \cref{lemma:quad_ratio}}
  \end{align*}
  %
  Plugging in this bound and \cref{lemma:hale} into \cref{eqn:pre_bound} completes the proof.
\end{proof}

\begin{corollary}
  Let $\bK$ and $\bb$ be the inputs to \cref{alg:ciq}, producing the output $\ba_J' \approx \bK^{-1/2} \bb$ after $J$ iterations with $Q$ quadrature points.
  The difference between $\ba_J$ and $\bK^{1/2} \bb$ is bounded by:
  %
  \begin{align*}
    \left\Vert \ba_J' - \bK^{-\frac 1 2} \bb \right\Vert_2
    &\leq
    \overbracket{
      \bigo{\frac{1}{\lambda_\text{min}} \exp\left( -\frac  {2 Q \pi^2}{\log \kappa(\bK) + 3} \right)}
    }^{\text{Quadrature error}}
    \\
    &\phantom{=} +
    \overbracket{
      \frac{ 2 \log \left( 5 \sqrt{\kappa(\bK)} \right)  \left( Q \lambda_\text{max} \right) }{\pi \sqrt{\lambda_\text{min}}}
      \left( \frac{ \sqrt{\kappa(\bK)} - 1}{ \sqrt{\kappa(\bK)} + 1} \right)^{J-1}
      \left\Vert \bb \right\Vert_2.
    }^{\text{msMINRES error}}
  \end{align*}
  %
  where $\lambda_\text{max}$ is the maximum eigenvalue of $\bK$, and $\kappa(\bK)$ is the condition number of $\bK$.
  \label{thm:ciq_convergence_inverse}
\end{corollary}

\begin{proof}
  Note that $\ba_J' = \bK^{-1} \ba_J$, where $\ba_J$ is the CIQ estimate of $\bK^{1/2} \bb$.
  Through Cauchy-Schwarz we have:
  \[
    \left\Vert \ba_J' - \bK^{-\frac 1 2} \bb \right\Vert_2
    \leq \left\Vert \bK^{-1} \right\Vert_2 \left\Vert \ba_J - \bK^{\frac 1 2} \bb \right\Vert_2
    = \frac{1}{\lambda_\text{min}} \left\Vert \ba_J - \bK^{\frac 1 2} \bb \right\Vert_2,
  \]
  where the final term is bounded by \cref{thm:ciq_convergence}.
\end{proof}
