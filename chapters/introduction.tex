\chapter{Introduction}
\label{chapt:introduction}

\section{Outline of Contributions}

%\paragraph{Preventing catastrophic failure.}

%\paragraph{Improved predictive pipelines.}

%\paragraph{Detecting anomalous inputs.}
%Machine learning models will only generalize to data that are sufficiently similar to the training data.
%If a model encounters \emph{out-of-distribution} (OOD) inputs -- inputs that deviate from the distribution of training data -- its predictions are likely to be erroneous or nonsensical \cite{begoli2019uncertainty,jiang2012calibrating}.
%%This may occur if the model is used in scenarios that experience covariate shift \cite{sugiyama2007covariate} or if the model encounters previously-unseen categories of data \cite{yu2017occ,hassen2018openset}.
%%Such scenarios are examples of \emph{out-of-distribution} (OOD) inputs.
%This phenomenon is illustrated in \cref{fig:ood_teaser}, which displays predictions from a neural network trained to predict prices of middle-class houses in Kentucky.
%The model is able to make sensible predictions on other Kentucky houses (left and center-left images).
%At the same time, the model vastly underestimates the price of a California mansion (center-right) and predicts and absurdly large price for a chair (right), as these inputs are not similar to any of the training set inputs.
%This is because the range of predicted prices for the out-of-distribution matches the range of Kentucky housing prices.
%A practitioner would see that the predictions are well within the model's expected output values and would be unaware that these predictions are nonsensical given the supplied inputs.

%Well-modeled uncertainty estimates can to identify potentially anomalous data and prevent such erroneous predictions.
%In this example, \gp{finish}

%\paragraph{A principled exploration/exploitation tradeoff.}

%\paragraph{Interpretability and trustworthiness.}
%Good uncertainty estimates can provide a valuable extra bit of information to users of machine learning models when predictions may otherwise be difficult to interpret.
%As machine learning algorithms become increasingly complex, they also appear more ``black box'' to users of such systems.
%This presents several challenges, especially for models that are used to aid human decision makers in domains such as medicine, finance, and policy \gp{cite}.

%For such circumstances it is therefore desirable for predictions to be understandable or interpretable \gp{cite LIME, saliency, etc.}.
%There are many definitions for what constitutes a good ``explanation'' of black-box predictions, coming from policy makers \gp{cite gdpr, etc} and the research community \gp{cite} alike.
%Though there are disagreements between these various sources, a well-calibrated uncertainty estimate is typically seen as a bare-minimum requirement for an interpretable prediction \gp{cite}.
%Most humans -- even if they are unable to perform simple inferences \cite{gigerenzer2003simple} -- have natural intuition for interpreting confidence estimates as event-occurrence frequencies \cite{cosmides1996humans,hoffrage1998using}.
%Therefore, well-calibrated confidence estimates from ML models can be easily interpreted by users.
%Moreover, the presence of uncertainty estimates can affect a user's trust in a machine learning model.
%In a study by \citet{zhou2017effects}, humans were asked to plan a budget for construction tasks based on information provided by a machine learning model.
%Some participants received both predictions and confidence intervals from the machine learning model, while other users only received the predictions.
%On tasks with low-to-moderate cognitive overhead, participants who saw uncertainty scores reported higher levels of trust in the machine learning model.
