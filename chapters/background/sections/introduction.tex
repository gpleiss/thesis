For a majority of this thesis we will assume that we are building predictive models for supervised regression problems.
As a running example, imagine that we wish to predict a child's future weight $y \in \reals$ from some features about the child $\bx \in \reals^\numdim$---such as the their current weight, their mother's income level, etc.
We assume that the child's height $y$ can be explained by 1) some latent function $f(\bx)$ of the features and 2) some observational noise $\epsilon$:
%
\begin{equation}
  y = f \left( \bx \right) + \epsilon, \quad \epsilon \sim \normaldist{0}{\sigma_\text{obs}^2}.
  \label{eqn:regression}
\end{equation}
%
In particular, we assume the observational noise follows some Gaussian distribution with variance $\sigma_\text{obs}^2$.
At a high level, the goal of supervised regression is to find a function $f(\bx)$ (or a distribution of functions) that fits the available training data $\dset_\text{train} = \{(\bx_1, y_1), \ldots, (\bx_n, y_n)\}$ and makes accurate predictions on new data points $( \bxtest, \ytest )$ at test time.

This setup is one of the most studied problems in machine learning, and there are numerous different approaches to tackling this problem.
In this thesis we concern ourselves with the approach of \emph{Gaussian process regression}, which is a {\bf non-parametric} and {\bf Bayesian} approach to regression.
It is non-parametric because the class of possible functions $f(\cdot)$ is defined through relationships between pairs of training data pointsl
rather than through some prescribed functional form (e.g. linear functions, quadratic functions, neural networks, etc.)
It is Bayesian because its predictions come from the \emph{distribution} of all possible functions $f(\cdot)$.
In other words, the choice of $f(\cdot)$ is marginalized out.
