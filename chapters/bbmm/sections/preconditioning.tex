\section{Preconditioning}
\label{sec:preconditioning}

While each iteration of mBCG performs large parallel matrix-matrix operations that utilize hardware efficiently, the iterations themselves are sequential.
A natural goal for better utilizing hardware is to trade off fewer sequential steps for slightly more effort per step.
We accomplish this goal using \emph{preconditioning} \cite{golub2012matrix,saad2003iterative,demmel1997applied,van2003iterative}, which introduces a matrix $\bP$ to solve the related linear system
\begin{equation*}
  \bP^{-1} \trainK \bc = \bP^{-1}\by
\end{equation*}
instead of $\trainK^{-1} \by$.
Both systems are guaranteed to have the same solution, but the preconditioned system's convergence depends on the conditioning of $\bP^{-1}\trainK$ rather than that of $\trainK$.

\citet{cutajar2016preconditioning} explores using preconditioned conjugate gradients for exact GP training, where they use various sparse GP methods (as well as some classical methods) as preconditioners.
However, the methods in \citet{cutajar2016preconditioning} do not provide general purpose preconditioners.
For example, methods like Jacobi preconditioning have no effect when using a stationary kernel \cite{cutajar2016preconditioning,wilson2015thoughts}, and many other preconditioners have $\bigomega{N^{2}}$ complexity, which dominates the complexity of most scalable GP methods.

We observe two requirements of a preconditioner to be used in general for GP training.
First, in order to ensure that preconditioning operations do not dominate running time when using scalable GP methods, the preconditioner should afford roughly linear time solves and space.
Second, we should be able to efficiently compute the log determinant of the preconditioner matrix, $\log \vert \bP \vert$.
This is because the mBCG algorithm applied to the preconditioned system estimates $\log \vert \bP^{-1}\trainK \vert$ rather than $\log \vert \trainK \vert$. We must therefore compute
$
  \label{eq:logdet_adjusted}
  \log \vert \trainK \vert = \log \vert \bP^{-1}\trainK \vert + \log \vert \bP \vert.
$

For one possible preconditioner, we turn to the {\bf partial pivoted Cholesky decomposition} \cite{harbrecht2012low}.
The pivoted Cholesky algorithm allows us to compute a low-rank approximation of a positive definite matrix, $\bK_{\bX\bX} \approx \bL_{k} \bL_{k}^{\top}$.
We precondition mBCG with $(\bL_k \bL_k^\top + \sigma^2 \bI)^{-1}$, where $\sigma^2$ is the Gaussian likelihood's noise term.
Intuitively, if $\bP_{k}=\bL_{k}\bL_{k}^{\top}$ is a good approximation of $\bK_{\bX\bX}$, then $(\bP_{k} + \sigma^{2}\bI)^{-1}\trainK \approx \bI$.

Unlike the standard Cholesky decomposition, which computes an exact factorization in $N$ iterations,
the partial pivoted Cholesky decomposition produces a \emph{rank-R} factorization in $R \ll N$ iterations.
Therefore, it does not has the same asymptotic or GPU-acceleration concerns as the standard factorization.
Before describing the preconditioner and its effect on mBCG convergence, we will first introduce the pivoted Cholesky algorithm and its properties.


\subsection{The Pivoted Cholesky Decomposition}

In this section, we review a full derivation of the \emph{pivoted Cholesky decomposition} as used for preconditioning in our paper.
To begin, observe that the standard Cholesky decomposition can be seen as producing a sequentially more accurate low rank approximation to the input matrix $\bA$.
In particular, the Cholesky decomposition algorithm seeks to decompose a matrix $\bA$ as:
%
\begin{equation}
 \label{eq:cholesky_full}
 \begin{bmatrix} \bA^{(11)} & \bA^{(12)} \\ \bA^{(12)^\top} & \bA^{(22)}\end{bmatrix}
 =
 \begin{bmatrix} \bL^{(11)} & \bzero \\ \bL^{(12)} & \bL^{(22)}\end{bmatrix}
 \begin{bmatrix} \bL^{(11)^\top} & \bL^{(12)^\top} \\ \bzero & \bL^{(22)^\top}\end{bmatrix}
\end{equation}
%
%Note that that $\bA^{(11)} = \bL^{(11)} \bL^{(11)^\top}$, $\bA^{(12)} = \bL^{(11)} \bL^{(21)^\top}$, and $\bA^{(22)} = \bL^{(21)} \bL^{(21)^\top} + \bL^{(22)} \bL^{(22)^\top}$.
From these equations, we can obtain $\bL^{(11)}$ by recursively Cholesky decomposing $\bA^{(11)}$, $\bL^{(21)^\top}$ by computing $\bL^{(21)^\top} = \bL^{(11)^{-1}} \bA^{(12)}$,
and finally $\bL^{(22)}$ by Cholesky decomposing the Schur complement $\bS_2 = \bA^{(22)} - \bL^{(21)} \bL^{(21)^\top}$.

We can view each iteration of the Cholesky decomposition as producing a slightly higher rank approximation to the matrix $\bA$.
In particular, if  $ \bA = \left[ K^{(11)}, \:\: \bb^{\top}; \:\:\:\: \bb, \:\: \bA^{(22)} \right], $
then $\bL^{(11)} = \sqrt{K^{(11)}}$, $\bL^{(21)} = \left( 1 / \sqrt{K^{(11)}} \right) \bb$, and the Schur complement is $\bS_2 = \bA^{(22)} - \left(1/ K^{(11)} \right) \bb \bb^{\top}$.
Therefore:
%
\begin{align}
  \bA &=
    \begin{bmatrix} \sqrt{ K^{(11)} } \\ \frac{1}{\sqrt{K^{(11)}}} \bb \end{bmatrix}
    \begin{bmatrix} \sqrt{ K^{(11)} } & \frac{1}{\sqrt{K^{(11)}}} \bb^\top \end{bmatrix}
    +
    \begin{bmatrix} 0 & 0 \\ 0 & \bS_2 \end{bmatrix}
  \nonumber \\
  &= \bv_{1}\bv_{1}^{\top} + \begin{bmatrix} 0 & 0 \\ 0 & \bS_2 \end{bmatrix}
  \label{eq:chol_incomplete}.
\end{align}
%
Because the Schur complement is positive definite, we can continue by recursing on the $(N-1) \times (N - 1)$ Schur complement $\bS_2$ to get another vector.
In particular, if $\bS_2 = \bv_{2}\bv_{2}^{\top} + \left[0, \:\: 0; \:\:\:\: 0, \:\: \bS_3 \right]$, then:
%
\begin{align}
  \bA  = \bv_{1}\bv_{1}^{\top} + \begin{bmatrix} 0 \\ \bv_{2} \end{bmatrix} \begin{bmatrix} 0 & \bv_2^\top \end{bmatrix} +
  \begin{bmatrix} 0 & 0 \\ 0 & \bS_3 \end{bmatrix}
\end{align}
%
In general, after $R$ iterations, defining $\hat{\bv}_{i} = \left[ \bzero; \:\: \bv_i \right]$ we obtain
%
\begin{align}
  \bA  = \sum_{i=1}^{R} \hat{\bv}_{i} \hat{\bv}_{i}^{\top} + \begin{bmatrix} \bzero & \bzero \\ \bzero & \bS_{R+1} \end{bmatrix}
\end{align}
%
The matrix $\bP_{R} = \sum_{i=1}^{R}\hat{\bv}_{i}\hat{\bv}_{i}^{\top}$ can be viewed as a rank-$R$ approximation to $\bA$, with
$$
  \Vert \bA - \bP_{R} \Vert_{2} = \Vert \bS_{R+1} \Vert_2.
$$

To improve the accuracy of the low rank approximation, one natural goal is to minimize the norm of the Schur complement, $\Vert \bS_{i} \Vert$, at each iteration.
\citet{harbrecht2012low} suggest to permute the rows and columns of $\bS_{i}$ (with $\bS_{1} = \bA$) so that the upper-leftmost entry in $\bS_{i}$ is the maximum diagonal element.
In the first step, this amounts to replacing $\bA$ with $\bPi_{1} \bA \bPi_{1}^\top$, where $\bPi_{1}$ is a permutation matrix that swaps the first row and column with whichever row and column corresponds to the maximum diagonal element of $\bA$.
Thus:
%
\begin{align*}
  \bPi_{1} \bA \bPi_{1}^\top = \bv_{1}\bv_{1}^{\top} + \begin{bmatrix} 0 & 0 \\ 0 & \bS_2 \end{bmatrix}.
\end{align*}
%
To proceed, one can apply the same pivoting rule to $S$ to achieve $\pi_{2}$:
%
\begin{align*}
  \left( \bPi_{2} \bPi_{1} \right) \bA \left( \bPi_{1}^\top \bPi_{2}^\top \right) = \bPi_{2} \bv_{1}\bv_{1}^{\top} \bPi_2^\top
  + \hat \bv_2 \hat \bv_2^\top + \begin{bmatrix} \bzero & \bzero \\ \bzero & \bS_3 \end{bmatrix}.
\end{align*}
%
In general, after $R$ steps, we obtain:
\begin{align}
  \bA = \sum_{i=1}^{R} \left( \prod_{j=1}^i \bPi_j \right) \hat{\bv}_{i} \hat{\bv}_{i} \left( \prod_{j=1}^i \bPi_{i-j+1}^\top \right)
  + \begin{bmatrix} \bzero & \bzero \\ \bzero & \bS_{R+1} \end{bmatrix}.
\end{align}
Collecting the $\left( \prod_{j=1}^i \bPi_j \right) \hat{\bv}_{i}$ vectors in to a matrix $\bL_R$ gives us a rank-$R$ approximation of $\bA$:
%
\begin{equation}
  \bA = \bL_{R} \bL_{R}^{\top} + \begin{bmatrix} \bzero & \bzero \\ \bzero & \bS_{R+1} \end{bmatrix} \approx \bL_{R} \bL_{R}^{\top}.
\end{equation}
%
$\bL_{R} \in \reals^{N \times R}$ is referred to as the {\bf partial pivoted Cholesky factor} of $\bA$.

\subsection{Pivoted Cholesky as a Preconditioner}

Let $\bL_{R} \in \reals^{N \times R}$ be the rank-$R$ pivoted Cholesky factor of $\bK_{\bX\bX}$.
To accelerate mBCG for GP training, we introduce the {\bf partial pivoted Cholesky preconditioner} for the matrix $\trainK = \bK_{\bX\bX} + \sigma^2_\text{obs} \bI$
\begin{equation}
  \trainP_R = \bL_{R} \bL_{R}^\top + \sigma^2_\text{obs} \bI.
\end{equation}
%
There are three key properties of our preconditioner that we wish to highlight:
%
\begin{observation}[Properties of the rank-$R$ pivoted Cholesky preconditioner.]
  {\ }
  \begin{enumerate}
    \item $\bL_{R}$ can be computed in $\bigo{\row{\trainK}R^{2}}$ time, where $\row{\trainK}$ is the time required to retrieve a single row of $\trainK$.

    \item Linear solves with $\trainP_R = \bL_{R} \bL_{R}^{\top} + \sigma_\text{obs}^{2} \bI$ can be performed in $\bigo{NR^{2}}$ time using the Woodbury matrix formula.\footnote{
      The Woodbury matrix formula is a $\bigo{R^2 N}$ formula for ``rank-R plus diagonal'' solves:
      $$\left( \bL_{R} \bL_{R}^{\top} + \sigma^{2}_\text{obs} \bI \right)^{-1} \bb = {\sigma_\text{obs}^{-2}}\bb - {\sigma_\text{obs}^{-4}} \: \bL_{R} \left( \bI - {\sigma^{-2}_\text{obs}} \bL_{R}^{\top} \bL_{R} \right)^{-1} \bL_{R}^{\top}\bb.$$
    }

    \item The log determinant of $\trainP_R$ can be computed in $\bigo{NR^{2}}$ time using the matrix determinant lemma.\footnote{
      The matrix determinant lemma is an analog of the Woodbury formula for determinants:
      $$\log \vert \bL_{R} \bL_{R}^\top + \sigma^{2}_\text{obs} \bI \vert = \log \vert \bI - {\sigma^{-2}} \bL_{R}^{\top} \bL_{R} \vert + 2 N \log \sigma_\text{obs}.$$
    }
  \end{enumerate}
\end{observation}
For a standard positive definite matrix, \citet{harbrecht2012low} observes that $\row{\trainK} = \bigo{N}$ running time.
In general this is also true for low-rank and structured matrices that we don't wish to explicitly compute, this computation time remains roughly linear.
Therefore, assuming $R \ll N$ (for example, $R \approx 5$, computing and using $\trainP$ is less expensive than a single matrix multiplication with $\trainK$.

More important than its runtime however are its convergence properties.
In general, the low-rank nature of $\trainP$ is an ideal choice for many kernel matrices with rapidly decaying spectra.
Such kernels tend to be horribly conditioned yet are well approximated by a low rank matrix.
We will demonstrate in \cref{sec:results} that $\trainP$ dramatically improves the convergence of mBCG for a wide variety of kernels.
Below, we will prove some theoretical results for certain classes of kernels.

\gp{Stopped here.}

\paragraph{Theoretical analysis.}
For univariate RBF kernels, it is possible to show that the conditioning of mBCG improves \emph{super-exponentially} with the rank of partial pivoted Cholesky preconditioner.
%
\begin{lemma}
  \label{thm:condition_number}
  Let $\bK_{\bX\bX} \in \reals^{N \times N}$ be a univariate RBF kernel matrix, and let $\bL_{R}$ be its rank-$R$ pivoted Cholesky factor.
  Then there exists a constant $B > 0$ so that the condition number $\kappa(\trainP^{-1}\trainK)$ satisfies the following bound:
  \begin{equation}
    \kappa \left( \trainP_{k}^{-1}\trainK \right)
    \triangleq \left\Vert \trainP_{k}^{-1}\trainK \right\Vert_{2} \left\Vert \trainK^{-1}\trainP_{k} \right\Vert_{2}
    \leq \Bigl( 1 + \bigo{N e^{-BR}} \Bigr)^2.
  \end{equation}
\end{lemma}
%
\noindent
(See \cref{app:convergence} for a proof).
Using \cref{thm:condition_number}

\begin{theorem}[Convergence of pivoted Cholesky-preconditioned CG]
  \label{thm:cg_convergence_rbf}
  Let $\bK_{\bX\bX} \in \reals^{n \times n}$ be a $n \times n$ univariate RBF kernel, and let $\bL_k \bL_k^\top$ be its rank $k$ pivoted Cholesky decomposition.
  Assume we are using preconditioned CG to solve the system $\trainK^{-1} \by = (\bK_{\bX\bX} + \sigma^2 \bI)^{-1} \by$ with preconditioner $\trainP = (\bL_k \bL_k^\top + \sigma^2 \bI)$.
  Let $\bc_J$ be the $J^\textrm{th}$ solution of CG, and let $\bc^{*} = \trainK^{-1} \by$ be the exact solution.
  Then there exists some $b > 0$ such that:
  \begin{equation}
    \Vert \bc^{*} - \bc_{J} \Vert_{\trainK}
    \leq 2 \left(1/(1 + \bigo{\exp(kb)/n}\right)^{J} \left\Vert \bc^{*} - \bc_{0} \right\Vert_{\trainK}.
  \end{equation}
\end{theorem}
%
\cref{thm:cg_convergence_rbf} implies that we should expect the convergence of conjugate gradients to improve \emph{exponentially} with the rank of the pivoted Cholesky decomposition used for RBF kernels. In our experiments we observe significantly improved convergence for other kernels as well (\cref{sec:results}). Furthermore, we can leverage \cref{thm:condition_number} and existing theory from \cite{ubaru2017fast} to argue that preconditioning improves our log determinant estimate. In particular, we restate Theorem 4.1 of \citet{ubaru2017fast} here:
\begin{theorem}[Theorem 4.1 of \citet{ubaru2017fast}]
  \label{thm:slq_convergence}
  Let $\bK_{\bX\bX} \in \reals^{n \times n}$, and let $\bL_k \bL_k^\top$ be its rank $k$ pivoted Cholesky decomposition.
  Suppose we run $p \geq \frac{1}{4} \sqrt{ \kappa \left( \trainP_{k}^{-1}\trainK \right) } \log \frac{D}{\epsilon}$ iterations of mBCG,
  where $D$ is a term involving this same condition number that vanishes as $k \to n$ (see \cite{ubaru2017fast}),
  and we use $t \geq \frac{24}{\epsilon^{2}}\log(2/\delta)$ vectors $\bz^{(i)}$ for the solves.
  Let $\Gamma$ be the log determinant estimate from \eqref{eq:slq}. Then:
  \begin{equation}
    \textrm{Pr}\left[\vert \log \vert \trainP^{-1}\trainK \vert - \Gamma \vert \leq \epsilon\vert \log \vert \trainP^{-1}\trainK \vert \vert \right] \geq 1 - \delta.
  \end{equation}
\end{theorem}
Because \cref{thm:condition_number} states that the condition number $\kappa \left( \trainP_{k}^{-1}\trainK \right)$ decays exponentially with the rank of $\bL_{k}$, \cref{thm:slq_convergence} implies that we should expect that the number of CG iterations required to accurately estimate $\log \vert \trainP^{-1}\trainK \vert$ decreases quickly as $k$ increases.
In addition, in the limit as $k \rightarrow n$ we have that $\log \vert \trainK \vert = \log \vert \trainP \vert$.
This is because $\log \vert \trainP^{-1}\trainK \vert \rightarrow 0$ (since $\trainP^{-1}\trainK$ converges to $\bI$) and we have that $\log \vert \trainK \vert = \log \vert \trainP^{-1}\trainK \vert + \log \vert \trainP \vert$.
Since our calculation of $\log \vert \trainP \vert$ is exact, our final estimate of $\log \vert \trainK \vert$ becomes more exact as $k$ increases.
In future work we hope to derive a more general result that covers multivariate settings and other kernels.
%
\citet{harbrecht2012low} explores the use of the pivoted Cholesky decomposition as a low rank approximation, although primarily in a scientific computing context.
\citet{bach2013sharp} considers using random column sampling as well as the pivoted Cholesky decomposition as a low-rank approximation to kernel matrices.
However, \citet{bach2013sharp} treats this decomposition as an approximate training method, whereas we use the pivoted Cholesky decomposition primarily
as a preconditioner, which avoids any loss of accuracy from the low rank approximation as well as the complexity of computing derivatives.
