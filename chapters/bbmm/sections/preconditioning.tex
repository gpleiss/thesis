\label{sec:preconditioning}

While each iteration of mBCG performs large parallel matrix$\times$matrix operations that utilize hardware efficiently, the iterations themselves are sequential.
A natural goal for better utilizing hardware is to trade off fewer sequential steps for slightly more effort per step.
We accomplish this goal using \emph{preconditioning} \citep[e.g.][]{demmel1997applied,saad2003iterative,van2003iterative,golub2012matrix}, which introduces a matrix $\bP$ to solve the related linear system
\begin{equation}
  \left( \trainP^{- \frac 1 2} \trainK \trainP^{- \frac 1 2}\right) \bC = \trainP^{- \frac 1 2} \left[ \by, \:\:, \bz^{(1)}, \:\:, \ldots, \:\: \bz^{(T)} \right].
  \label{eqn:precond_system}
\end{equation}
instead of $\trainK^{-1} \left[  \by, \:\:, \bz^{(1)}, \:\:, \ldots, \:\: \bz^{(T)} \right].$.
Both systems are guaranteed to have the same solution, but the preconditioned system's convergence depends on the conditioning of $\trainP^{- \frac 1 2} \trainK \trainP^{- \frac 1 2}$ rather than that of $\trainK$.
Despite the matrix square roots in \cref{eqn:precond_system}, preconditioned CG only needs access to $\trainP$ and its inverse $\trainP^{-1}$ (see \cref{alg:std_pcg}).



\subsection{Modifying mBCG for Preconditioning}
\label{sec:precond_requirements}
We have to make some special adjustments to our modified batch CG algorithm in order to use preconditioning.
While the preconditioned system simply returns $\trainK^{-1} \by$, we need to modify the random probe vectors $\bz^{(1)}$, $\ldots$, $\bz^{(T)}$ to get correct estimates of $\log \vert \trainK \vert$ and $\tr{ \trainK^{-1} ( \partial \trainK / \partial \btheta) }$.
In particular, we will perform the solves
%
\begin{equation}
  \label{eqn:mod_cg_call_precond}
  \trainK^{-1} \left[ \by, \:\: \bz^{(1)}, \:\: \cdots, \:\: \bz^{(T)} \right], \quad \bz^{(i)} \sim \normaldist{\bzero}{\trainP}.
\end{equation}

The only difference between \cref{eqn:mod_cg_call_precond} and the original set of solves in \cref{eqn:mod_cg_call}
is that the $\bz^{(i)}$ probe vectors have covariance $\Ev{ \bz^{(i)} \bz^{(i)^\top} } = \trainP$ (rather than unit covariance).
To understand why this is the case, recall from \cref{sec:method_bbmm} that our log determinant estimate is given by:
%
\begin{align*}
	\log \left\vert \trainK \right\vert
	&\approx \Evover{\bz^{(i)} \sim \normaldist{0}{\bI} }{\bz^{(i)^\top} \bQ^{(i)} \left( \log \bT^{(i)} \right) \bQ^{(i)^\top} \bz^{(i)}},
\end{align*}
%
where $\bT^{(1)}$, $\ldots$, $\bT^{(T)}$---the Lanczos tridiagonal matrices corresponding to $\trainK$ and probe vectors $\bz^{(i)}$, $\ldots$, $\bz^{(T)}$---are computed from byproducts of the CG iterations.
However, if we precondition mBCG with $\trainP$, then the $\bT^{(i)}$ matrices will instead correspond to the \emph{preconditioned system}
$(\trainP^{- \frac 1 2} \trainK \trainP^{- \frac 1 2})$ and \emph{preconditioned probe vectors} $(\trainP^{- \frac 1 2} \bz^{(i)})$.
Consequentially, the stochastic Lanczos quadrature estimate will actually return
%
\begin{align}
	\log \left\vert \trainP^{- \frac 1 2} \trainK \trainP^{- \frac 1 2} \right\vert
	&\approx \!
	\Evover{\bz^{(i)} \sim \normaldist{0}{\trainP} }{
		\left( \bz^{(i)^\top} \trainP^{-\frac 1 2} \right) \!
		\bQ^{(i)} \! \left( \log \bT^{(i)} \right) \! \bQ^{(i)^\top}
		\!\! \left( \trainP^{-\frac 1 2} \bz^{(i)} \right)
	}.
	\label{eqn:slq_precond}
\end{align}
%
\cref{eqn:slq_precond} motivates the use of $\bz^{(i)} \sim \normaldist{\bzero}{\trainP}$ as the probe vectors:
the resulting preconditioned vectors $\trainP^{- \frac 1 2} \bz^{(i)}$ will be samples from $\normaldist{\bzero}{\bI}$ and can therefore be used for a stochastic trace estimate.

\paragraph{Estimating $\log \vert \trainK \vert$ and $\tr{ \trainK^{-1} (\partial \trainK / \partial \btheta) }$ from preconditioned mBCG.}
To compute $\log \vert \trainK \vert$ from \cref{eqn:slq_precond}, we note that
\[
\log \vert \trainK \vert = \log \vert \trainP^{- \frac 1 2} \trainK \trainP^{- \frac 1 2} \vert + \log \vert \trainP \vert.
\]
We therefore estimate the first using stochastic Lanczos quadrature (\cref{eqn:slq_precond}) and then ``correct'' this estimate with the log determinant of the preconditioner.
Note that this will still be an unbiased estimate of $\log \vert \trainK \vert$ if we can exactly compute $\log \vert \trainP \vert$ (in an efficient manner).

To estimate $\tr{ \trainK^{-1} (\partial \trainK / \partial \btheta) }$ from the new probe vectors $\bz^{(i)} \sim \normaldist{\bzero}{\trainP}$,
we note that we can form a stochastic trace estimate from the following:
\begin{align}
	\tr{\trainK^{-1}\frac{\partial \trainK}{\partial \btheta}}
	&=
	\tr{
		\trainK^{-1}\frac{\partial \trainK}{\partial \btheta}
		\Evover{\bz^{(i)} \sim \normaldist{0}{\trainP} }{
			\trainP^{-1} \bz^{(i)} \bz^{(i)^\top}
		}
	}
  \nonumber \\
	&\approx
	\Evover{\bz^{(i)} \sim \normaldist{0}{\trainP} }{
		\left( \bz^{(i)} \trainK^{-1} \right)
		\left( \frac{\partial \trainK}{\partial \btheta} \: \trainP^{-1} \bz^{(i)} \right)
	}.
	\label{eqn:trace_est_precond}
\end{align}
%
The only differences between \cref{eqn:trace_est_precond} and the non-preconditioned trace estimate in \cref{eqn:trace_deriv_estimate} are
\begin{enumerate*}
	\item we use $\bz^{(i)} \sim \normaldist{\bzero}{\trainP}$ as probe vectors, and
	\item the derivative term $\partial \trainK / \partial \btheta$ is applied to the preconditioned vectors $\trainP^{-1} \bz^{(i)}$.
\end{enumerate*}

\paragraph{Requirements of mBCG preconditioners.}
Based on the above discussion, we observe three requirements of any preconditioner $\trainP$ that can be used with mBCG for GP training.
First, in order to ensure that preconditioning operations do not dominate running time in \cref{alg:mod_pcg}, the preconditioner should afford roughly linear-time solves and linear space.
%\footnote{
	%Recall from \cref{alg:std_pcg} that preconditioned conjugate gradients only requires computing $\trainP^{-1}$ and $\trainP$, not $\trainP^{- \frac 1 2}.$
%}
Second, we should be able to efficiently compute the log determinant of the preconditioner matrix $\log \vert \trainP \vert$ to ``correct'' the log determinant estimate in \cref{eqn:slq_precond}.
Finally, we should be able to efficiently sample probe vectors $\bz^{(i)}$ from the distribution $\normaldist{\bzero}{\trainP}$.



\subsection{The Partial Pivoted Cholesky Preconditioner for mBCG}
\label{sec:piv_chol_precond}

For one possible preconditioner, we turn to the {\bf partial pivoted Cholesky decomposition} (as introduced in \cref{sec:piv_chol}).
The pivoted Cholesky algorithm allows us to compute a rank-$R$ approximation ($R \ll N$) of the training kernel matrix $\bK_{\bX\bX} \approx \bar\bL_{R} \bar\bL_{R}^{\top}$.
Our mBCG preconditioner will be
\begin{equation}
  \trainP_R = \bar\bL_{R} \bar\bL_{R}^\top + \sigma^2_\text{obs} \bI,
\end{equation}
where $\sigma_\text{obs}^2$ is the Gaussian likelihood's noise term.
Intuitively, if $\bar\bL_{R}\bar\bL_{R}^{\top}$ is a good low-rank approximation of $\bK_{\bX\bX}$, then $(\bar\bL_{R} \bar\bL_R^\top + \sigma_\text{obs}^{2}\bI)^{-1}\trainK \approx \bI$.


Unlike the standard Cholesky decomposition, which computes an exact factorization in $N$ iterations,
the partial pivoted Cholesky decomposition produces a \emph{rank-R} factorization in $R \ll N$ iterations, and therefore does not share its asymptotic concerns.
Moreover, it meets the requirements outlined in \cref{sec:precond_requirements}:
%
\begin{observation}[Properties of the rank-$R$ pivoted Cholesky preconditioner]
  {\ }
  \begin{enumerate}
    \item $\bar\bL_{R}$ can be computed in $\bigo{\row{\trainK}R^{2}}$ time, where $\row{\trainK}$ is the time required to retrieve a single row of $\trainK$
      (see \cref{sec:piv_chol}).

    \item Storing $\bar\bL_{R}$ requires $\bigo{NR}$ memory.

    \item Linear solves with $\trainP_R = \bar\bL_{R} \bar\bL_{R}^{\top} + \sigma_\text{obs}^{2} \bI$ can be performed in $\bigo{NR^{2}}$ time using the Woodbury matrix formula.\footnote{
      The Woodbury matrix formula is a $\bigo{R^2 N}$ formula for ``rank-R plus diagonal'' solves:
      $$\left( \bar\bL_{R} \bar\bL_{R}^{\top} + \sigma^{2}_\text{obs} \bI \right)^{-1} \bb = {\sigma_\text{obs}^{-2}}\bb - {\sigma_\text{obs}^{-4}} \: \bar\bL_{R} \left( \bI - {\sigma^{-2}_\text{obs}} \bar\bL_{R}^{\top} \bar\bL_{R} \right)^{-1} \bar\bL_{R}^{\top}\bb.$$
    }

    \item The log determinant of $\trainP_R$ can be computed in $\bigo{NR^{2}}$ time using the matrix determinant lemma.\footnote{
      The matrix determinant lemma is an analog of the Woodbury formula for determinants:
      $$\log \vert \bar\bL_{R} \bar\bL_{R}^\top + \sigma^{2}_\text{obs} \bI \vert = \log \vert \bI - {\sigma_\text{obs}^{-2}} \bar\bL_{R}^{\top} \bar\bL_{R} \vert + 2 N \log \sigma_\text{obs}.$$
    }

  \item Samples from $\normaldist{ \bzero}{ \trainP_R }$ can be drawn with $\bigo{NR^2}$ computation using the reparameterization trick \cite{kingma2014auto}.\footnote{
      Draw standard normal vectors $\bepsilon_1 \in \reals^R$ and $\bepsilon_2 \in \reals^N$.
      By the reparameterization trick, $\left( \bar\bL_R \bepsilon_1 + \sigma_\text{obs} \bepsilon_2 \right)$ is a sample from $\normaldist{ \bzero}{( \bar\bL_R \bar\bL_R^\top + \sigma^2_\text{obs} \bI )}$.
    }
  \end{enumerate}
\end{observation}
%
\noindent
Assuming $R \ll N$ (for example, $R \approx 5$), computing and using $\trainP$ is less expensive than a single matrix multiplication with $\trainK$.
While the $R \approx 5$ iterations required to compute $\bar\bL_R$ are inherently sequential, we note that this is far fewer iterations than the standard Cholesky factorization.
This large reduction in sequential computation makes this preconditioner rather amenable to GPU acceleration.

Perhaps more important than its runtime are its convergence properties.
In general, the low-rank nature of $\trainP$ is an ideal choice for many kernel matrices with rapidly decaying spectra.
Such kernels tend to be horribly conditioned yet are well approximated by a low rank matrix, making $\trainP$ a natural choice to precondition.
We empirically demonstrate in \cref{sec:bbmm_results} that $\trainP$ dramatically improves the convergence of mBCG for a wide variety of kernels.
Below, we will prove some theoretical results for certain classes of kernels.


\paragraph{Theoretical analysis.}
Kernels with rapidly decaying eigenvalues (i.e. kernels that are well approximated by low-rank matrices) will see the largest improvements with the pivoted Cholesky preconditioner.
Based on the work of \citet{harbrecht2012low}, one can prove the following lemma about kernel condition numbers:
%
\begin{lemma}
  \label{thm:condition_number}
  Let $\bar\bL_{R}$ be the rank-$R$ pivoted Cholesky factor of kernel matrix $\bK_{\bX\bX} \in \reals^{N \times N}$.
  If the first $R$ eigenvalues $\lambda_1$, $\ldots$, $\lambda_R$ of $\bK_{\bX\bX}$ satisfy
	\begin{equation}
		4^{i}\lambda_{i} \leq \bigo{e^{-Bi}}, \quad i \in \{ 1, \:\: \ldots, \:\: R \},
		\label{eqn:pcp_condition}
	\end{equation}
	for some $B>0$, then the condition number $\kappa(\trainP^{-1}\trainK) \triangleq \Vert \trainP_{k}^{-1}\trainK \Vert_{2} \Vert \trainK^{-1}\trainP_{k} \Vert_{2}$
	satisfies the following bound:
  \begin{align}
    \kappa \left( \trainP^{-1}\trainK \right)
    &\leq \Bigl( 1 + \bigo{\sigma^{-2}_\text{obs} N e^{-BR}} \Bigr)^2
		\nonumber
  \end{align}
	where $\trainP = \left( \bar\bL_R \bar\bL_R^\top + \sigma^2_\text{obs} \bI \right)$ and $\trainK = \left( \bK_{\bX\bX} + \sigma^2_\text{obs} \bI \right)$.
\end{lemma}
%
It so happens that the exponentially-decaying eigenvalue assumption actually holds for certain classes of kernels.
For example, the eigenvalues of univariate RBF kernels are guaranteed to decay \emph{super-exponentially} (see \cref{app:univariate_rbf}).
In our experiments we observe significantly improved conditioning for other kernels as well (\cref{sec:bbmm_results}).

Using \cref{thm:condition_number}, we can prove the following statements about the solves/log determinant estimates from preconditioned mBCG:
%
\begin{theorem}[Convergence of solves from preconditioned mBCG]
  \label{thm:precond_mbcg_solves}
  Let $\bK_{\bX\bX} \in \reals^{N \times N}$ be a $N \times N$ kernel that satisfies the eigenvalue condition of \cref{eqn:pcp_condition},
	and let $\bar\bL_R$ be its rank-$R$ pivoted Cholesky factor.
	After $J$ iterations of mBCG with preconditioner $\trainP = (\bar\bL_R \bar\bL_R^\top + \sigma_\text{obs}^2 \bI)$,
	the difference between $\bc_J$ and true solution $\trainK^{-1} \by$ is bounded by:
	%
  \begin{equation*}
    \left \Vert \trainK^{-1} \by - \bc_{J} \right \Vert_{\trainK}
    \leq \Bigg[ \frac 1 {1 + \bigo{\sigma^{2}_\text{obs} e^{RB}/N}} \Bigg]^{J}
    \left \Vert \trainK^{-1} \by \right \Vert_{\trainK},
		\nonumber
  \end{equation*}
	%
	where $\trainK = (\bK_{\bX\bX} + \sigma^2_\text{obs} \bI)$ and $B > 0$ is a constant.
\end{theorem}
%
\begin{theorem}[Convergence of log determinants from preconditioned mBCG]
  \label{thm:precond_mbcg_logdet}
  Assume $\bK_{\bX\bX} \in \reals^{N \times N}$ satisfies the eigenvalue condition of \cref{eqn:pcp_condition}.
	Suppose we estimate $\Gamma \approx \log \vert \trainP^{-1} \trainK \vert$ using \cref{eqn:slq_precond} with:
	\begin{itemize}
		\item $J \geq \mathcal{O} \left[ (1 + \sigma^{-2}_\text{obs} N e^{-BR}) \log \left( ( 1 + \sigma^{-2}_\text{obs} N e^{-BR} ) / \epsilon \right) \right]$ iterations of mBCG (for some constant $B > 0$), and
		\item $T \geq \frac{32}{\epsilon^2} \log \left( \frac 2 \delta \right)$ random $\bz^{(i)} \sim \normaldist{\bzero}{\trainP}$ vectors.
	\end{itemize}
  Then the error of the stochastic Lanczos quadrature estimate $\Gamma$ is probabilistically bounded by:
  \begin{equation*}
    \textrm{Pr}\left[\Bigl\vert \log \vert \trainP^{-1} \trainK \vert - \Gamma \Bigr\vert \leq \epsilon N \right] \geq \left( 1 - \delta \right).
  \end{equation*}
\end{theorem}
%
\cref{thm:precond_mbcg_solves} implies that the mBCG solves---used to compute both $\trainK^{-1} \by$ and $\tr{ \trainK^{-1} (\partial \trainK / \partial \btheta)}$---will converge \emph{exponentially} quicker as the rank of the pivoted Cholesky decomposition increases.
\cref{thm:precond_mbcg_logdet} implies that the number of iterations needed to accurately estimate $\log \vert \trainP^{-1}\trainK \vert$ also decreases quickly as $R$ increases.
Furthermore, in the limit as $R \rightarrow N$ we have that $\log \vert \trainK \vert = \log \vert \trainP \vert$.
%This is because $\log \vert \trainP^{-1}\trainK \vert \rightarrow 0$ (since $\trainP^{-1}\trainK$ converges to $\bI$) and we have that $\log \vert \trainK \vert = \log \vert \trainP^{-1}\trainK \vert + \log \vert \trainP \vert$.
Since our calculation of $\log \vert \trainP \vert$ is exact, the final estimate of $\log \vert \trainK \vert$ has less stochasticity as $R$ increases.

\paragraph{Related work.}
\citet{cutajar2016preconditioning} explore preconditioned conjugate gradients for GP training, where they use various sparse GP methods (as well as some classical methods) as preconditioners.
However, the methods described in \citet{cutajar2016preconditioning} are not general purpose preconditioners.
For example, methods like Jacobi preconditioning have no effect when using a stationary kernel \cite{wilson2015thoughts}, and many other preconditioners have $\bigomega{N^{2}}$ complexity, which dominates the complexity of most scalable GP methods.

\citet{bach2013sharp} uses the pivoted Cholesky decomposition as a low-rank approximation to kernel matrices.
However, \citet{bach2013sharp} treats this decomposition as an approximate training method, whereas we use the decomposition primarily as a preconditioner and thus avoid any loss of accuracy from the low rank approximation.


