%!TEX root=../main.tex
\section{Adapting BBMM and LOVE to Large-Scale Exact GPs}
\label{sec:largeexact_method}

The BBMM and LOVE methods presented in \cref{chapter:bbmm,chapter:love} effectively utilize GPU acceleration and reduce the time complexity of GPs to $\bigo{N^2}$.
As we will demonstrate, these advantages make it possible to scale exact GPs to very large datasets---up to two orders of magnitude larger than what is possible with Cholesky-based training/inference \cite{nguyen2019exact}.

In this section, we make slight modifications to the BBMM and LOVE algorithms for large-scale GPs.
Firstly, we reduce the MVM memory requirements to $\bigo{N}$ using partitioned and distributed kernel matrices.
Secondly, we offer practical guidelines to speed up BBMM convergence for large datasets.




\subsection{Reducing Memory Requiremnts to $\mathcal{O}(N)$}

The primary input to the modified batched conjugate gradients (mBCG) algorithm of \cref{sec:bbmm_method} is {\tt mvm\_$\trainK$}, a black-box function that performs MVMs using the kernel matrix $\trainK$.
Each iteration of mBCG updates four (batches) of vectors: $\bc$ (the current solution), $\br$ (the current error), $\bd$ (the ``search'' direction for the next solution), and $\bz$ (a preconditioned error term)---see \cref{alg:std_pcg,alg:mod_pcg} for details.
Storing these vectors requires exactly $4N$ space.
In addtion, the mBCG algorithm stores the tridiagonal matrices for stochastic Lanczos quadrature, which requires $\bigo{J}$ memory for $J$ iterations of mBCG.
Therefore, the quadratic space cost associated with MVM-based exact GPs comes from storing the matrix $\trainK$.

Typically, {\tt mvm\_$\trainK$} is implemented by first computing the full $N \times N$ kernel matrix $\trainK$.
Although forming $\trainK$ requires $\bigo{N^2}$ memory, the \emph{output} of the MVM $\trainK \bb$ requires only $\bigo{N}$ memory.
By using a \emph{map-reduce style algorithm} for this computation, we can reduce the memory requirement of {\tt mvm\_$\trainK$}.

\paragraph{Partitioned kernel MVMs.}
We first partition the data matrix $\bX \in \reals^{N \times D}$ into $P$ partitions, each of which contains roughly $N/P$ data points:
%
\[
  \bX = \begin{bmatrix}
    \bX^{(1)}; &
    \cdots &
    ; \bX^{(P)}
  \end{bmatrix}
\]
%
where we use ``$;$'' to denote row-wise concatenation.
For each $\bX^{(\ell)}$, we can compute $\widehat \bK_{\bX^{(\ell)}\!\bX}$, which is a roughly $(N / P) \times N$ kernel matrix.
We can now rewrite the training kernel matrix as a concatenation of the $P$ partitions:
%
\[
  \widehat \bK_{\bX\!\bX} = \begin{bmatrix}
      \widehat \bK_{\bX^{(1)}\bX}; &
    \cdots &
    ; \widehat \bK_{\bX^{(P)}\bX}
  \end{bmatrix}.
\]
%
Computing each partition requires access to the full training set $\bX$, which we assume fits in memory.
However, each partition $\widehat \bK_{\bX^{(\ell)}\!\bX}$ contains only $1/P$ of the entries of the full kernel matrix.
Rewriting the matrix-vector product $\trainK \bb$ in terms of these partitions:
%
\[
  \widehat \bK_{\bX\!\bX} \bb = \begin{bmatrix}
    \widehat \bK_{\bX^{(1)}\!\bX} \bb; &
    \cdots &
    ; \widehat \bK_{\bX^{(P)}\bX} \bb
  \end{bmatrix},
\]
%
we see that this MVM can be computed in smaller components by separately computing each $\widehat \bK_{\bX^{(\ell)}\!\bX} \bb$ and concatenating the results.
We discard each kernel partition $\widehat \bK_{\bX^{(\ell)}\!\bX}$ once its MVM has been computed.
This MVM only allocates new memory to temporarily store the $(N / P) \times N$ kernel matrix partition $\widehat \bK_{\bX^{(\ell)}\!\bX}$.
The other memory requirements (i.e. storing $\bX$ and $\bb$) are linear in $N$.

This algorithm allows us to reduce memory usage in exchange for additional yet easily parallelizable computations.
If $P=1$ then we have the na\"ive $\bigo{N^2}$ memory MVM procedure. As $P \to N$, mBCG will only require $\bigo{N}$ memory.

(After developing our own custom partitioned-MVM approach, we were made aware of the KeOps software package \cite{charlier2020kernel} which uses low-level GPU code to speed up partitioned MVMs.
With this library, partitioned MVMs can even be faster than standard MVMs.)


\paragraph{Distributed parallel MVMs.}
MVM-based training/inference can easily take advantage of multiple GPUs or distributed computational resources.
Each MVM partition $\widehat \bK_{\bX^{(l)}\!\bX} \bb$ can be performed on a different device.
Additionally, we note that distributed/parallel MVMs require only $O(N)$ communication between devices (i.e. the memory required to share $\bX$ and the partial output vector $\widehat \bK_{\bX^{(l)}\!\bX} \bb$).
In contrast, distributing the Cholesky decomposition requires $\bigo{N^2}$ communication \citep{nguyen2019exact}.






\subsection{Practical Considerations}


\paragraph{Preconditioning.}
As in \cref{chapter:bbmm}, we use the partial pivoted Cholesky preconditioner for training and computing predictive means.
In \cref{chapter:bbmm} the preconditioner size is typically limited to $R=5$; however, we find that preconditioners of size $R=100$ provide a noticeable speed improvement for large GPs.
Computing a rank $R$ partial pivoted Cholesky preconditioner requires only $R$ kernel matrix rows: a $\bigo{N}$ space dependence.
While each partitioned MVM computes the kernel sub-matrices from scratch, the preconditioner is computed only once.
Therefore, for large values of $N$ it can be efficient to increase the size of the preconditioner to reduce the number of mBCG iterations.


\paragraph{mBCG convergence criteria.}
Importantly, mBCG is \emph{not} an approximate method for performing linear solves.
Rather, it is a method that performs solves to a specified tolerance.
If this tolerance is sufficiently tight, solve are exact up to machine precision.
Thus, it is analogous to using gradient descent for convex optimization problems (which is in fact what is happening).

At test time, we find that (nearly) exact solves $\widehat \bK_{\bX \! \bX}^{-1} \by$  are critical for good predictive means.
Therefore, we set the convergence criterion of CG to be $\Vert \trainK \bc - \by \Vert_2 / \Vert \by \Vert_2 \leq 0.001$, where $\bc$ is the solution from mBCG.
For hyperparameter optimization, we find that less strict tolerances are surprisingly sufficient.
A looser convergence criterion of up to $\Vert \trainK \bc - \by \Vert_2 / \Vert \by \Vert_2 = 1$ has little impact on final model performance.


\paragraph{Pre-training.}
On large datasets, each training iteration (i.e. each call to mBCG) may take several minutes.
We can reduce the total number of training iterations by first initializing the hyperparameters to sensible defaults.
As a simple initialization strategy, we pre-train exact Gaussian process models on a subset of training data (e.g. $N=10,\!000$).
After this initialization, only a few iterations (i.e. $< 5$ gradient descent steps) are necessary on the full dataset (see \cref{fig:initialization}).
