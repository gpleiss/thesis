%!TEX root=../main.tex
\section{Adapting BBMM and LOVE to Large-Scale Exact GPs}
\label{sec:largeexact_method}

The BBMM and LOVE methods presented in \cref{chapter:bbmm,chapter:love} effectively utilize GPU acceleration and reduce the time complexity of GP training/predictions to $\bigo{N^2}$.
As we will demonstrate, these advantages make it possible to scale exact GPs to very large datasets---up to two orders of magnitude than what is possible with Cholesky-based training/inference \cite{nguyen2019exact}.

In this section, we make slight modifications to the BBMM and LOVE algorithms for large-scale exact GPs.
Firstly, we reduce the memory requirements of MVMs to $\bigo{N}$ using partitioned and distributed kernel matrices.
In addition, we offer practical guidelines to speed up BBMM for large datasets.




\subsection{Reducing Memory Requiremnts to $\mathcal{O}(N)$}

The primary input to the modified batched conjugate gradients (mBCG) algorithm of \cref{sec:bbmm_method} is {\tt mvm\_$\trainK$}, a black-box function that performs MVMs using the kernel matrix $\trainK$.
Besides the storage cost associated with {\tt mvm\_$\trainK$}, each iteration of mBCG updates four (batches) of vectors: $\bc$ (the current solution), $\br$ (the current error), $\bd$ (the ``search'' direction for the next solution), and $\bz$ (a preconditioned error term).
(See \cref{alg:mod_pcg}.)
Storing these vectors requires exactly $4N$ space.
In addtion, the mBCG algorithm stores the tridiagonal matrices for stochastic Lanczos quadrature, which requires $\bigo{J}$ memory for $J$ iterations of mBCG.
The quadratic space cost associated with MVM-based exact GPs comes from storing the matrix $\trainK$.

Typically, {\tt mvm\_$\trainK$} is implemented by first computing the full $N \times N$ kernel matrix $\trainK$, then computing the matrix-vector product with the full matrix.
Although forming $\trainK$ requires $\bigo{N^2}$ memory, the \emph{output} of the MVM $\trainK \bb$ requires only $\bigo{N}$ memory.
By using a \emph{map-reduce style algorithm} to compute $\trainK \bb$, we can reduce the memory requirement of {\tt mvm\_$\trainK$} to \emph{linear in $N$}.

\paragraph{Partitioned kernel MVMs.}
To compute $\trainK \bb$ in a map-reduce fashion, we partition the kernel matrix $\trainK$ such that \textit{we only store a constant number of rows at any given time}.
We first partition the data matrix $\bX \in \reals^{N \times D}$ into $P$ partitions, each of which contains roughly $N/P$ data points:
%
\[
  \bX = \begin{bmatrix}
    \bX^{(1)}; &
    \cdots &
    ; \bX^{(P)}
  \end{bmatrix}
\]
%
where we use ``$;$'' to denote row-wise concatenation.
For each $\bX^{(\ell)}$, we can compute $\widehat \bK_{\bX^{(\ell)}\!\bX}$, which is a roughly $(N / P) \times N$ kernel matrix.
We can now rewrite the training kernel matrix as a concatenation of the $P$ partitions:
%
\[
  \widehat \bK_{\bX\!\bX} = \begin{bmatrix}
      \widehat \bK_{\bX^{(1)}\bX}; &
    \cdots &
    ; \widehat \bK_{\bX^{(P)}\bX}
  \end{bmatrix}.
\]
%
Computing each partition requires access to the full training set $\bX$, which we assume fits in memory.
However, each partition $\widehat \bK_{\bX^{(\ell)}\!\bX}$ contains only $1/P$ of the entries of the full kernel matrix.
Rewriting the matrix-vector product $\trainK \bb$ in terms of these partitions:
%
\[
  \widehat \bK_{\bX\!\bX} \bb = \begin{bmatrix}
    \widehat \bK_{\bX^{(1)}\!\bX} \bb; &
    \cdots &
    ; \widehat \bK_{\bX^{(P)}\bX} \bb
  \end{bmatrix},
\]
%
we see that this MVM can be computed in smaller components by separately computing each $\widehat \bK_{\bX^{(\ell)}\!\bX} \bb$ and concatenating the results.
We discard each kernel partition $\widehat \bK_{\bX^{(\ell)}\!\bX}$ once its MVM has been computed.
This MVM only allocates new memory to temporarily store the $(N / P) \times N$ kernel matrix partition $\widehat \bK_{\bX^{(\ell)}\!\bX}$.
The other memory requirements (i.e. storing $\bX$ and $\bb$) are linear in $N$.

This algorithm allows us to reduce memory usage in exchange for sequential but easily parallelizable computations.
If $P=1$ then we have the na\"ive $\bigo{N^2}$ memory MVM procedure. As $P \to N$, PCG will only require $\bigo{N}$ memory.
By keeping a partition in memory only until its component of the MVM has been computed, we can train GPs with an $\bigo{N}$ memory requirement.

After developing our own custom partitioned-MVM approach, we were made aware of the KeOps software package \cite{charlier2020kernel} which computes map-reduced kernel-vector products.
This library contains a very efficient GPU implementation in which partitioned MVMs can even be faster than their non-partitioned counterparts.


\paragraph{Distributed parallel MVMs.}
MVM-based training/inference can easily take advantage of multiple GPUs or distributed computational resources.
Each MVM partition $\widehat \bK_{\bX^{(l)}\!\bX} \bb$ can be performed on a different device.
Thus we can compute partitioned MVMs in parallel to attain wall-clock speedups proportional to the number of devices available.

Additionally, we note that distributed parallel MVMs require only $O(N)$ communication.
Each partitioned matrix multiplication only has to supply each device with a new right-hand-side vector $\bb$.
Finally, if $W$ devices are used, the output from each device will be a vector of length $N / PW$.
Thus only $\bigo{N}$ memory is copied to or from the devices.
In contrast, distributing the Cholesky decomposition requires $\bigo{N^2}$ communication \citep{nguyen2019exact}.





\subsection{Practical Considerations}

\paragraph{Predictions with large-scale GPs.}
At test time, we must compute the predictive mean and variance given in \cref{eqn:predictive_mean} and \cref{eqn:predictive_var}.
As described in \cref{sec:motivation_means}, predictive means require the inner product between $\bk_{\bX \bxtest}$ and the cached vector $\blue \ba = \trainK^{-1} \by$.
Similarly, the LOVE method of \cref{chapter:love} generates a cachable low-rank approximation ${\color{blue} \bR^\top \bR} \approx {\color{blue} \trainK^{-1}}$ for predictive variances.
After these caches have been computed, predictive distributions can be computed through $\bigo{N}$ vector inner products---an especially fast operation on GPUs.
In practice, we observe that both the predictive mean and variance can be computed in less than a second on a single GPU regardless of dataset size.


\paragraph{Preconditioning.}
As in \cref{chapter:bbmm}, we use the partial pivoted Cholesky preconditioner when training and computing predictive means.
In \cref{chapter:bbmm} the preconditioner size is typically limited to $R=5$; however, we find that preconditioners of up to size $R=100$ provide a noticeable wall-clock speed improvement for large exact GPs.
Computing a rank $R$ partial pivoted Cholesky preconditioner requires only $R$ kernel matrix rows: an already $\bigo{N}$ space dependence.
While each iteration of CG requires computing each kernel matrix partition from scratch, the preconditioner is computed only once.
Therefore, for large values of $N$ it can be efficient to increase the size of the preconditioner to an extent if it reduces the number of CG iterations.


\paragraph{mBCG convergence criteria.}
Importantly, mBCG is \emph{not} an approximate method for performing linear solves.
Rather, it is a method that consumes time to perform solves to a specified tolerance.
If this tolerance is low, the solve is exact up to machine precision.
Thus, it is analogous to using gradient descent to solve a convex optimization problem (which is in fact what is happening).

At test time, we find that (nearly) exact solves $\widehat \bK_{\bX \! \bX}^{-1} \by$  are critical for good predictive means.
Therefore, we set the convergence criterion of CG to be $\Vert \trainK^{-1} \bc - \by \Vert_2 / \Vert \by \Vert_2 \leq 0.001$, where $\bc$ is the solution from mBCG.
For hyperparameter training, however, we find that, interestingly, less strict convergence criteria suffice.
A looser convergence criterion of up to $\Vert \trainK^{-1} \bc - \by \Vert_2 / \Vert \by \Vert_2 = 1$ has little impact on final model performance.
Given that predictions using our approach are highly efficient (see \cref{tab:large_exact_gp_timings}), it may be interesting to investigate alternative approximate methods for finding good hyperparameters, and then using the techniques in this chapter for exact inference and predicitons.
