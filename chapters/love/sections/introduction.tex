%!TEX root=../main.tex
\section{Introduction}
Gaussian processes (GPs) are fully probabilistic models which can naturally estimate predictive uncertainty through posterior variances.
These uncertainties play a pivotal role in many application domains.
For example, uncertainty information is crucial when incorrect predictions could have catastrophic consequences, such as in medicine \cite{schulam2017if} or large-scale robotics \cite{deisenroth2015gaussian};
Bayesian optimization approaches typically incorporate model uncertainty when choosing actions \cite{snoek2012practical,deisenroth2011pilco,wang2017max};
and reliable uncertainty estimates are arguably useful for establishing trust in predictive models,
especially when predictions would be otherwise difficult to interpret
\cite{doshi2017roadmap,zhou2017effects}.

\emph{Although predictive uncertainties are a primary advantage of GP models, they have recently become their computational bottleneck.}
Historically, the use of GPs has been limited to problems with small datasets, since learning and inference computations na\"ively scale cubically with the number of data points ($n$).
Recent advances in \emph{inducing point methods} have managed to scale up GP training and computing predictive means to larger datasets \cite{snelson2006sparse,quinonero2005unifying,titsias2009variational}.
\emph{Kernel Interpolation for Scalable Structured GPs} (KISS-GP) is one such method that scales to millions of data points \cite{wilson2015kernel,wilson2015thoughts}.
For a test point $\bxtest$, KISS-GP expresses the predictive mean as $\bw_{\bxtest}^\top {\blue \ba'}$, where $\blue \ba'$ is a pre-computed vector dependent only on training data, and $\bw_\bxtest$ is a sparse interpolation vector.
This formulation affords the ability to compute predictive means in \emph{constant time}, independent of $n$.

However, these computational savings do not extend naturally to predictive uncertainties.
With KISS-GP, computing the predictive covariance between two points requires $\bigo{n + m \log m}$ computations, where $m$ is the number of inducing points.
While this asymptotic complexity is lower than standard GP inference and alternative scalable approaches, it becomes prohibitive when $n$ is large, or when making many repeated computations.
Additionally, drawing samples from the predictive distributions -- a necessary operation in many applications -- is similarly expensive.
Existing fast approximations for these operations \cite{papandreou2011efficient,wilson2015thoughts,wang2017max} typically incur a significant amount of error.
Matching the reduced complexity of predictive mean inference without sacrificing accuracy has remained an open problem.

In this paper, we provide a solution based on the tridiagonalization algorithm of \citet{lanczos1950iteration}.
Our method takes inspiration from KISS-GP's mean computations: we express the predictive covariance between $\bx^*_i$ and $\bx^{*}_j$ as
$\bw_{\bxtest_i}^\top \blue{\bC} \: \bw_{\bxtest_j}$,
where $\blue C$ is an $M \times M$ matrix dependent only on training data.
However, we take advantage of the fact that $\blue \bC$ affords fast matrix-vector multiplications (MVMs) and avoid explicitly computing the matrix.
Using the Lanczos algorithm, we can efficiently decompose $\blue \bC$ as two rank-$J$ matrices $\blue \bC \approx \blue{\bR \bR^{\top \prime}}$ in \emph{nearly linear time}.
After this one-time upfront computation, and due to the special structure of $\blue {\bR,\bR^\prime}$, all variances can be computed in \emph{constant time} -- $\bigo{J}$ -- per (co)variance entry.
We extend this method to sample from the predictive distribution at $T$ points in $\bigo{T + M}$ time -- independent of training dataset size.

We refer to this method as LanczOs Variance Estimates, or LOVE{} for short.\footnote{
  LOVE{} is implemented in the GPyTorch library.
  Examples are available at \url{http://bit.ly/gpytorch-examples}.
}
LOVE{} has the lowest asymptotic complexity for computing predictive (co)variances and drawing samples with GPs.
We empirically validate LOVE{} on seven datasets and find that it consistently provides substantial speedups over existing methods \emph{without sacrificing accuracy}.
Variances and samples are accurate to within four decimals, and can be computed \emph{up to 18,000 times faster.}
