\section{Motivation}
\label{sec:love_motivation}

\cref{eqn:predictive_mean} and \cref{eqn:predictive_var} describe how to compute predictive means and (co)-variances (restated below, assuming a prior mean of $\bzero$ for brevity):
%
\begin{align}
  \meantest{\bxtest}
  &= \bk_{\bX \bxtest}^\top {\color{blue} \trainK^{-1} \by}
  \label{eqn:predictive_mean_2}
  \\
  \covtest{\bxtest}{\bxtestprime}
  &= k(\bxtest, \bxtestprime) - \bk_{\bX\bxtest}^\top {\color{blue} \trainK^{-1}} \bk_{\bX\bxtestprime} + \sigma^2_\text{obs}.
  \label{eqn:predictive_covar_2}
\end{align}
%
The terms in {\color{blue} blue} represent terms that only depend on the training data.
If we are using a GP model to make a single prediction, then computing both of these terms requires a single call to conjugate gradients (${\color{blue} \trainK^{-1}} \bk^\top_{\bX\bxtest}$).
However, if we are making predictions on thousands of test points, these repeated CG calls may become prohibitively expensive, even when using the mBCG algorithm to parallelize the solves.
Since the expensive computations in \cref{eqn:predictive_mean_2,eqn:predictive_covar_2} primarily involve the training data terms, our goal is to \emph{pre-compute and cache} the most computationally intensive parts of these equations.
After this pre-computation, subsequent predictions ideally should be computationally cheap---i.e. $\bigo{N}$ or less.

\subsection{Computing Predictive Means}
\label{sec:motivation_means}

Before discussing predictive (co)-variances, which will be the primary focus of this chapter, we will first discuss a pre-computation strategy for predictive means:
Since the ${\color{blue} \trainK^{-1} \by \triangleq \ba}$ term only depends on training data, it can be cached and re-used for all predictive means.
Each subsequent predictive mean is simply the inner product between the $\bk_{\bX \bxtest}$ vector and the pre-computed $\color{blue} \ba$ vector, which in general takes $\bigo{N}$ time.

This cost can be reduced even further for KISS-GP models, as discussed in \cref{sec:kissgp} and \citep{wilson2015thoughts}.
Recall that KISS-GP approximates the training and test covariances as:
%
\begin{align*}
  \tilde{\bk}_{\bX \bxtest} = {\color{blue} \bW^\top_{\bX} \bK_{\bZ \bZ}} \bw_{\bxtest},
  \quad
  {\color{blue} \tilde{\bK}_{\bX \bX}} = {\color{blue} \bW^\top_{\bX} \bK_{\bZ \bZ} \bW_{\bX}},
\end{align*}
%
where $\bK_{\bZ\bZ} \in \reals^{M \times M}$ is the Toeplitz inducing kernel matrix, $\bW_\bX \in \reals{M \times N}$ is the sparse interpolation for training points $X$, and $\bw_{\bxtest} \in \reals{M}$ is the sparse interpolation for $\bxtest$.
Plugging these approximations into \cref{eqn:predictive_mean_2} gives us
\[
  \meantest{\bxtest} = \bw_{\bxtest}^\top {\blue{\underbracket{\bK_{\bZ\bZ} \bW_{\bX}(\bW_{\bX}^{\top}\bK_{\bZ\bZ}\bW_{\bX} + \sigma_\text{obs}^{2} \bI)^{-1} \by}_{\ba'}}}.
\]
where again the blue terms only depend on training data.
After pre-computing the $\ba'$ vector, all subsequent means are the inner product between $\color{blue} \ba'$ and the \emph{$\bigo{1}$ sparse} $\bw_\bxtest$ vector.
These computations are $\bigo{1}$ because of this sparsity.

\subsection{Computing (Co)-Variances without Pre-computation}

The predictive (co)-variances are more challenging, as the only term in \cref{eqn:predictive_covar_2} that does not depend on test data is $\blue \trainK^{-1}$.
A common pre-computation is to form the Cholesky factorization $\blue \bL \bL^\top = \trainK$ ($\bigo{N^3}$ time, $\bigo{N^2}$ memory).
After factorization, all subsequent solves take $\bigo{N^2}$ time.
However, this cubic dependence on $N$ and quadratic memory may be infeasible for $N \geq 10,\!000$.

It is possible to obtain some computational savings when using inducing point methods.
For example, if we replace $\bk_{\bX \bxtest}$ and $\blue \trainK$ with their corresponding KISS-GP approximations in \cref{eqn:predictive_covar_2}, then we have:
%
\begin{align}
  \covtest{\bxtest}{\bxtestprime}
  &\approx
  \sigma^2_\text{obs} + k(\bxtest, \bxtestprime)
  \nonumber \\
  &\phantom{\approx} \:\: - \bw_{\bxtest}^{\top} {\color{blue}\underbracket{{\bK_{\bZ\bZ} \bW_{\bX} \left( \bW^\top_{\bX} \bK_{\bZ\bZ} \bW_\bX + \sigma^2_\text{obs} \bI \right)^{-1} \bW^\top_{\bX} \bK_{\bZ \bZ}}}_{\blue \bC}} \bw_{\bxtestprime}.
  \label{eqn:pred_covar_approx}
\end{align}
%
$\blue \bC$, the braced portion of \cref{eqn:pred_covar_approx}, does not depend on the test points $\bxtest_{i}$, $\bxtest_{j}$ and therefore can be pre-computed during training.
The primary cost of this pre-computation is the $M$ solves with $\blue (\bW_\bX^\top {\bK}_{\bZ\bZ} \bW_\bX + \sigma_\text{obs}^2 \bI)$: one for each column vector in $\blue \bW^{\top}_{\bX} \bK_{\bZ\bZ}$, each of which takes $\bigo{N + M\log M}$ time with CG (see \cref{sec:kissgp} or \citet{wilson2015kernel}).
The total time for this pre-computation is therefore $\bigo{MN + M^2 \log M}$.
After pre-computation, \cref{eqn:pred_covar_approx} becomes
%
\begin{align}
  \covtest{\bxtest}{\bxtestprime} &\approx k_{\bxtest \bxtestprime} - \bw_{\bxtest}^\top \: {\blue \bC} \: \bw_{\bxtestprime} + \sigma^2_\text{obs}
    \label{eqn:pred_covar_ski_naive}
\end{align}
As $\bw_{\bxtest}$ contains only four nonzero elements, the inner product of $\bw^{*}_{i}$ with $\blue \bC$ takes $\bigo{M}$ time.
Thus predictive covariances with \cref{eqn:pred_covar_ski_naive} are $\bigo{M}$ after pre-computation.

Although this technique offers computational savings over the Cholesky method, the quadratic dependence on $M$ in the pre-computation phase is a computational bottleneck.
In contrast, all other operations with KISS-GP require at most linear storage and near-linear time.
Indeed, one of the hallmarks of KISS-GP is the ability to use a very large number of inducing points $M = \Theta(N)$ so that kernel computations are nearly exact.

