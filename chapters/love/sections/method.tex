\section{LanczOs Variance Estimates (LOVE)}
\label{sec:love_method}

\input algorithms/love

We propose to overcome these limitations through an altered pre-computation step.
In particular, we approximate $\blue C$ in \eqref{eq:pred_covar_ski_naive} as a low rank matrix.
Letting $\blue R$ and $\blue{R'}$ be $k \times m$ matrices such that $\blue R^\top \blue{R'} \approx \blue C$, we rewrite \eqref{eq:pred_covar_ski_naive} as:
%
% \begin{align}
%   k_{f\mid\dset}(\bxtest_i, \bxtest_j) &\approx k_{\bxtest_i \bxtest_j} - \bw_{\bx^{*^\top}_i} \blue R^{\top}\blue{R'} \: \bw_{\bxtest_j}.\nonumber \\
%                                  &\approx k_{\bxtest_i \bxtest_j} - (\blue R \bw_{\bxtest_i})^\top (\blue{R'} \: \bw_{\bxtest_j}).
%     \label{eq:pred_covar_ski_fast}
% \end{align}
\begin{align}
  k_{f\mid\dset}(\bxtest_i, \bxtest_j) &\approx k_{\bxtest_i \bxtest_j} - (\blue R\bw^{*}_i )^{\top}\blue{R'} \: \bw^{*}_j.
    \label{eq:pred_covar_ski_fast}
\end{align}
%
Variance computations with \eqref{eq:pred_covar_ski_fast} take $\bigo{k}$ time
due to the sparsity of $\bw_{\bxtest_i}$ and $\bw_{\bxtest_j}$.
Recalling the Lanczos approximation
$
    (\tilde{K}_{\bX\bX} + \sigma^2 \bI)^{-1}\bb \approx Q_{k}T^{-1}_{k}Q_{k}^{\top}\bb
$
from \autoref{subsec:lanczos}, we can efficiently derive forms for $\blue R$ and $\blue{R'}$:
\begin{align*}
  \blue C &= \bK_{U \! U} W_{\bX} \underbrace{(\tilde{K}_{\bX\!\bX} + \sigma^2 \bI)^{-1}}_{\text{Apply Lanczos}} W^\top_{\bX} \bK_{U \! U} \\
         &\approx \bK_{UU} W_X (Q_k T_k^{-1} Q_k^\top) W_X^\top K_{UU} \\
         &= \underbrace{K_{UU} W_X Q_k}_{\blue R^\top} \underbrace{T_k^{-1} Q_k^\top W_X^\top \bK_{UU}}_{\blue{R'}}
\end{align*}
\gp{Reference lanczos stuff}
To compute $\blue R$ and $\blue{R'}$, we perform $k$ iterations of Lanczos to achieve $(\tilde{K}_{\bX\bX} + \sigma^2\bI)  \approx Q_{k}T_{k}Q_{k}^{\top}$ using the average column vector $\mathbf{b}=\frac{1}{m}W_X^\top \bK_{UU}\mathbf{1}$ as a probe vector.
This partial Lanczos decomposition requires $k$ MVMs with $(\tilde \bK_{XX} + \sigma^2 \bI)$ for a total of $\bigo{k n + k m \log m}$ time (because of the KISS-GP approximation).
$\blue R$ and $\blue{R'}$ are $m \times k$ matrices, and thus require $\bigo{mk}$ storage.

\paragraph{To compute $\blue R$,} we first multiply $W_{\bX}Q_{k}$, which takes $\bigo{kn}$ time due sparsity of $W_\bX$.
The result is a $m \times k$ matrix.
Since $\bK_{UU}$ has Toeplitz structure, the multiplication $\bK_{U\!U} (W_{\bX}Q_{k})$ takes $\bigo{km \log m}$ time \cite{saatcci2012scalable}.
Therefore, computing $\blue R$ takes $\bigo{kn + km\log m}$ total time.

\paragraph{To compute $\blue {R'}$,} note that $\blue{R'} = T^{-1}_{k}\blue R^{\top}$.
Since $\tilde \bK_{X\!X}$ is positive definite, $T$ is as well (by properties of the Lanczos algorithm).
We thus compute $T^{-1}_k \blue R^\top$ using the Cholesky decomposition of $T$.
Computing/using this decomposition takes $\bigo{km}$ time since $T$ is tridiagonal \cite{loan1999introduction}.

In total, the entire pre-computation phase takes only $\bigo{kn + km \log m}$ time.
This is the same amount of time of a single marginal likelihood computation.
We perform the pre-computation as part of the training procedure since none of the terms depend on test data.
Therefore, during test time all predictive variances can be computed in $\bigo{k}$ time using \eqref{eq:pred_covar_ski_fast}.
As stated in \autoref{subsec:lanczos}, $k$ depends on the conditioning of the matrix and not its size \cite{golub2012matrix}.
$k\leq100$ is sufficient for most matrices in practice \cite{golub2012matrix}, and therefore $k$ can be considered to be constant.
Running times are summarized in \autoref{tab:running_times}.




\section{LOVE with KISS-GP}

\input algorithms/love_kissgp

We propose to overcome these limitations through an altered pre-computation step.
In particular, we approximate $\blue C$ in \eqref{eq:pred_covar_ski_naive} as a low rank matrix.
Letting $\blue R$ and $\blue{R'}$ be $k \times m$ matrices such that $\blue R^\top \blue{R'} \approx \blue C$, we rewrite \eqref{eq:pred_covar_ski_naive} as:
%
% \begin{align}
%   k_{f\mid\dset}(\bxtest_i, \bxtest_j) &\approx k_{\bxtest_i \bxtest_j} - \bw_{\bx^{*^\top}_i} \blue R^{\top}\blue{R'} \: \bw_{\bxtest_j}.\nonumber \\
%                                  &\approx k_{\bxtest_i \bxtest_j} - (\blue R \bw_{\bxtest_i})^\top (\blue{R'} \: \bw_{\bxtest_j}).
%     \label{eq:pred_covar_ski_fast}
% \end{align}
\begin{align}
  k_{f\mid\dset}(\bxtest_i, \bxtest_j) &\approx k_{\bxtest_i \bxtest_j} - (\blue R\bw^{*}_i )^{\top}\blue{R'} \: \bw^{*}_j.
    \label{eq:pred_covar_ski_fast}
\end{align}
%
Variance computations with \eqref{eq:pred_covar_ski_fast} take $\bigo{k}$ time
due to the sparsity of $\bw_{\bxtest_i}$ and $\bw_{\bxtest_j}$.
Recalling the Lanczos approximation
$
    (\tilde{K}_{\bX\bX} + \sigma^2 \bI)^{-1}\bb \approx Q_{k}T^{-1}_{k}Q_{k}^{\top}\bb
$
from \autoref{subsec:lanczos}, we can efficiently derive forms for $\blue R$ and $\blue{R'}$:
\begin{align*}
  \blue C &= \bK_{U \! U} W_{\bX} \underbrace{(\tilde{K}_{\bX\!\bX} + \sigma^2 \bI)^{-1}}_{\text{Apply Lanczos}} W^\top_{\bX} \bK_{U \! U} \\
         &\approx \bK_{UU} W_X (Q_k T_k^{-1} Q_k^\top) W_X^\top K_{UU} \\
         &= \underbrace{K_{UU} W_X Q_k}_{\blue R^\top} \underbrace{T_k^{-1} Q_k^\top W_X^\top \bK_{UU}}_{\blue{R'}}
\end{align*}
\gp{Reference lanczos stuff}
To compute $\blue R$ and $\blue{R'}$, we perform $k$ iterations of Lanczos to achieve $(\tilde{K}_{\bX\bX} + \sigma^2\bI)  \approx Q_{k}T_{k}Q_{k}^{\top}$ using the average column vector $\mathbf{b}=\frac{1}{m}W_X^\top \bK_{UU}\mathbf{1}$ as a probe vector.
This partial Lanczos decomposition requires $k$ MVMs with $(\tilde \bK_{XX} + \sigma^2 \bI)$ for a total of $\bigo{k n + k m \log m}$ time (because of the KISS-GP approximation).
$\blue R$ and $\blue{R'}$ are $m \times k$ matrices, and thus require $\bigo{mk}$ storage.

\paragraph{To compute $\blue R$,} we first multiply $W_{\bX}Q_{k}$, which takes $\bigo{kn}$ time due sparsity of $W_\bX$.
The result is a $m \times k$ matrix.
Since $\bK_{UU}$ has Toeplitz structure, the multiplication $\bK_{U\!U} (W_{\bX}Q_{k})$ takes $\bigo{km \log m}$ time \cite{saatcci2012scalable}.
Therefore, computing $\blue R$ takes $\bigo{kn + km\log m}$ total time.

\paragraph{To compute $\blue {R'}$,} note that $\blue{R'} = T^{-1}_{k}\blue R^{\top}$.
Since $\tilde \bK_{X\!X}$ is positive definite, $T$ is as well (by properties of the Lanczos algorithm).
We thus compute $T^{-1}_k \blue R^\top$ using the Cholesky decomposition of $T$.
Computing/using this decomposition takes $\bigo{km}$ time since $T$ is tridiagonal \cite{loan1999introduction}.

In total, the entire pre-computation phase takes only $\bigo{kn + km \log m}$ time.
This is the same amount of time of a single marginal likelihood computation.
We perform the pre-computation as part of the training procedure since none of the terms depend on test data.
Therefore, during test time all predictive variances can be computed in $\bigo{k}$ time using \eqref{eq:pred_covar_ski_fast}.
As stated in \autoref{subsec:lanczos}, $k$ depends on the conditioning of the matrix and not its size \cite{golub2012matrix}.
$k\leq100$ is sufficient for most matrices in practice \cite{golub2012matrix}, and therefore $k$ can be considered to be constant.
Running times are summarized in \autoref{tab:running_times}.


\subsection{Predictive distribution sampling with LOVE{}}
\label{sec:sampling_method}

LOVE{} can also be used to compute predictive \emph{covariances} and operations involving the predictive covariance matrix.
Let $\bXtest = [\bxtest_1, \ldots, \bxtest_t]$ be a test set of $t$ points.
%In many applications \eqref{eq:pred_covar} will be used only to compute the predictive variance terms for each $\bxtest_i$,  i.e. $k_{f \mid \dset} (\bxtest_i, \bxtest_i)$.
To draw samples from $\bfntest \! \mid \! \dset$ --- the predictive function evaluated on $\bxtest_1, \ldots, \bxtest_t$, the cross-covariance terms (i.e. $k_{f \mid \dset} (\bxtest_i, \bxtest_j)$) are necessary in addition to the variance terms ($k_{f \mid \dset} (\bxtest_i, \bxtest_i)$).
Sampling GP posterior functions is a common operation.
In Bayesian optimization for example, several popular acquisition functions -- such as predictive entropy search \cite{hernandez2014predictive}, max-value entropy search \cite{wang2017max}, and knowledge gradient \cite{frazier2009knowledge} -- require posterior sampling.

However, posterior sampling is an expensive operation when querying at many test points.
The predictive distribution $\bfntest \! \mid \! \dset$ is multivariate normal with mean $\mu_{f \mid \dset} (\bX^*) \in \reals^t$ and covariance $k_{f \mid \dset} (\bX^*, \bX^*) \in \reals^{t \times t}$.
We sample $\bfntest \! \mid \! \dset$ by reparametrizing Gaussian noise samples $\mathbf{v} \sim \normaldist{0}{\bI{}^{t\!\times \!t}}$:
%
\begin{equation}
  \mu_{f \mid \dset} (\bX^*) + S \mathbf{v},
  \label{eq:sample}
\end{equation}
%
where $S$ is a matrix such that $S S^\top = k_{f \mid \dset} (\bX^*, \bX^*)$.
Typically $S S^\top$ is taken to be the Cholesky decomposition of $k_{f \mid \dset} (\bX^*, \bX^*)$.
Computing this decomposition incurs a $\bigo{t^3}$ cost on top of the $\bigo{t^2}$ covariance evaluations.
This may be costly, even with constant-time covariance computations.
Parametric approximations are often used instead of exact sampling \cite{deisenroth2011pilco}.

\paragraph{A Fast Low-Rank Sampling Matrix.} We use LOVE{} and KISS-GP to rewrite \eqref{eq:pred_covar_ski_fast} as
%
\begin{align}
  k_{f \mid \dset} (\bX^*, \bX^*)
  &\approx \bW^\top_{X^*} \blue{K_{UU}} W_{X^*} - (\blue \bR \bW_{X^*})^\top (\blue{\bR'} \bW_{\bXtest})
    \notag \\
    &= \bW^\top_{\bX^*} \blue{\left( \bK_{UU} - \bR^\top \bR' \right)} \bW_{X^*}.
    \label{eq:pred_covar_ski_interp_form12}
\end{align}
%
where $W_{X^*} = [\bw_{x^*_1}, \ldots, \bw_{x^*_n}]$ is the interpolation matrix for test points.
We have reduced the full covariance matrix to a test-independent term ($\blue{ K_{UU} - R^\top R' }$) that can be pre-computed.
We apply the Lanczos algorithm on this term during pre-computation to obtain a rank-$k$ approximation:
%
\begin{align}
  \blue{K_{UU} - R^\top R' \approx Q'_k T'_k Q_k^{\prime\top} }.
  \label{eqn: lancapprox}
\end{align}
%
This Lanczos decomposition requires $k$ matrix vector multiplies with $\blue{ K_{UU} - R^{\top}R' }$, each of which requires $\bigo{m \log m}$ time.
Substituting \eqref{eqn: lancapprox} into \eqref{eq:pred_covar_ski_interp_form12}, we get:
%
\begin{align}
  k_{f \mid \dset} (\bX^*, \bX^*) = W^\top_{X^*} \blue{Q'_{k}T'_{k}Q_{k}^{\prime\top}} W_{X^*}.
  \label{eq:sampling_pre_cholesky}
\end{align}
If we take the Cholesky decomposition of $T'_k = \!L L^\top$ (a $\bigo{k}$ operation since $T'_{k}$ is tridiagonal), we rewrite \eqref{eq:sampling_pre_cholesky} as
%
\begin{align}
  k_{f \mid \dset} (\bX^*, \bX^*)
  &\approx  W^\top_{X^*} \blue{\underbrace{Q'_k L}_{S}} \blue{\underbrace{L^\top Q_k^{\prime\top}}_{S^\top}} W_{X^*}.
    \label{eq:pred_covar_ski_interp_form}
\end{align}
%
Setting $\blue{ S= Q'_k L_{T_k} }$, we see that $k_{f \mid \dset} (\bX^*, \bX^*) = (W_X^\top \blue S)(W_X^\top \blue S)^{\top}$.
$\blue S \! \in \! \reals^{m \times k}$ can be pre-computed and cached since it does not depend on test data.
In total, this pre-computation takes $\bigo{k m \log m + m k^2}$ time in addition to what is required for fast variances.
To sample from the predictive distribution, we need to evaluate \eqref{eq:sample}, which involves multiplying $W^\top_{X^*} S \mathbf{v}$.
Multiplying $\mathbf{v}$ by $S$ requires $\bigo{mk}$ time, and finally multiplying by $W^{\top}_{X^{*}}$ takes $\bigo{tk}$ time.
Therefore, drawing $s$ samples (corresponding to $s$ different values of $\mathbf{v}$) takes $\bigo{sk(t + m)}$ time total during the testing phase (see \autoref{tab:running_times}) -- a \emph{linear} dependence on $t$.


\subsection{Extension to additive kernel compositions}
LOVE{} is applicable even when the KISS-GP approximation is used with an additive composition of kernels,
%
\begin{equation}
  \tilde{k}(\bx^{(i)}, \bx^{(j)}) =
  \bw^{(1)\top}_{\bx^{(i)}} K^{(1)}_{U \! U} \bw^{(1)}_{\bx^{(j)}} + \ldots + \bw^{(d)\top}_{\bx^{(i)}} K^{(d)}_{U \! U} \bw^{(d)}_{\bx^{(j)}}.
  \notag
\end{equation}
Additive structure has recently been a focus in several Bayesian optimization settings, since the cumulative regret of additive models depends linearly on the number of dimensions
\cite{kandasamy2015high,wang2017batched,gardner2017discovering,wang2017max}.
Additionally, deep kernel learning GPs \citep{wilson2016stochastic,wilson2016deep} typically uses sums of one-dimensional kernel functions.
To apply LOVE{}, we note that additive composition can be re-written as
%
\begin{equation}
  \tilde{k}(\bx^{(i)}, \bx^{(j)}) =
  \begin{bmatrix}
    \bw^{(1)}_{\bx^{(i)}} \\
    \vdots \\
    \bw^{(d)}_{\bx^{(i)}}
  \end{bmatrix}^\top
  \!
  \begin{bmatrix}
    K^{(1)}_{U \! U} & \!\! \ldots \!\! & 0 \\
    \vdots & \!\! \ddots \!\! & \vdots \\
    0 & \!\! \ldots \!\! & K^{(d)}_{U \! U}
  \end{bmatrix}
  \!
  \begin{bmatrix}
    \bw^{(1)}_{\bx^{(j)}} \\
    \vdots \\
    \bw^{(d)}_{\bx^{(j)}}
  \end{bmatrix}.
  \label{eq:multi_dimensional_kernel_block}
\end{equation}
%
The block matrices in \eqref{eq:multi_dimensional_kernel_block} are analogs of their 1-dimensional counterparts in \eqref{eq:ski}.
Therefore, we can directly apply \autoref{alg:fast_pred_var}, replacing $W_X$, $\bw_{\bxtest_i}$, $\bw_{\bxtest_j}$, and $K_{UU}$ with their block forms.
The block $\bw$ vectors are $\bigo{d}$-sparse, and therefore interpolation takes $\bigo{d}$ time.
MVMs with the block $K_{UU}$ matrix take $\bigo{dm\log m}$ time by exploiting the block-diagonal structure. With $d$ additive components, predictive variance computations cost only a factor $\bigo{d}$ more than their 1-dimensional counterparts.
